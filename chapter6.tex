\chapter{The Path to Enlightenment}
So far all of the examples have contained only the events and
variables needed.  But the ability to find events, and their
variables is required to be truly effective at programming
tagsets.  This chapter will explain several techniques that
will illustrate which events
may work for a solution, and the variables that are bundled in
those events.  In short, this chapter will show how to get 
tagsets to reveal themselves.

\index{Finding Events}
\index{Events!Finding}
\section{Finding Events}
So far, to keep things simple, a few event names have been explained as
though they are the only events needed.
But there are many more events than those few that have been
revealed.  Like the events in your day, the event requests generated
by ODS also vary.  It all depends on the job being run
and which output file is examined.  The events for the body
file will be different than the contents file which will 
be different from the stylesheet file.  A job with bygrouping
will be different than a job without it.  Footnotes, graphs,
notes, the procedure,  all of these things cause variations
in the event requests that ODS makes.  Similarly, a job with
By group processing will be different than a job without it.
A Proc Print job with By group processing will also be different
from other procedures that also do By group processing, like Proc
Tabulate or Proc Means.  

\index{Tagsets!event\_map}
\index{Tagsets!short\_map}
\index{Tagsets!tpl\_style\_map}
\index{Tagsets!text\_map}
\subsection{The Short\_map Tagset}
Thankfully a tagset can be written that will tell
what the events are and when they occur.  In fact
there are several tagsets already created for this purpose.
All but one of these tagsets generate XML.  Where the tag
names are the event names.  
There is no need to worry if you don't know anything about
XML.  The output will become self evident when you see it.
The main difference between
these tagsets are the attributes displayed on the tags.
The Super\_map tagset is the newest and most capable mapping
tagset.  Although it may take a little learning
to get up to speed.  By default Super\_map behaves just 
like Short\_map.  Short\_map output is the simplest and easiest
to understand of all the mapping tagsets.
The Event\_map tagset is fairly verbose and displays the 
most attributes.
Short\_map shows only the values, names, and labels and is
more than adequate most of the time.  Tpl\_style\_map is
more concerned with style names, and the names for ODS output
objects. Text\_map is meant as a learning tool but is not as
helpful as the others once events are understood.  
The mapping
tagsets are shown in table \ref{mapping_tagsets} on page \pageref{mapping_tagsets}.

\begin{table}\caption{Event Mapping Tagsets}
\label{mapping_tagsets}
\begin{tabular}{l|l} \\ \hline
Super\_map &  Extremely versatile, it has the same abilities as event\_map 
and short\_map as well as regular expression matching on event names 
and values.  It is extremely useful for hunting down values within events.\\
Event\_map &  A fairly verbose XML map \\
Short\_map &  The same Event\_map but less data \\
tpl\_style\_map & Displays more style and template information \\
text\_map & A formatted text map.  More human readable than xml \\
odsxrpcs & The most verbose of all tagsets.  
All attributes are shown as tags.\\
\end{tabular}
\end{table}

Using the short\_map tagset without running a procedure is a good first
start.
All that is needed is two ods statements as shown 
in figure \ref{event_map1}
on page \pageref{event_map1}.
The corresponding output is shown in figure \ref{event_map1_out}
on page \pageref{event_map1_out}.
What is important about the output is that all of the events shown
are the result of the ODS invocation and close.  

\begin{fvcode}{event_map1}{Creating an Empty Event Map}
ods tagsets.short_map file='test.xml';
ods tagsets.short_map close;
\end{fvcode}

\index{Events!doc}
\index{Events!doc\_title}
\index{Events!Start:}
\index{Events!Finish:}
Each tag in this output is the name of an event.  Using standard
XML format a the <doc> tag corresponds to the Start: section
of the doc event.  The </doc> corresponds to the Finish: section
of the doc event.  The <doc\_title/> tag means that this is
a non-scoping event that does not really have a start and finish.
If the event is defined with a start: and finish: that is still ok,
but both sections will be triggered one after the other when the
event occurs.

\index{Events!doc\_body}
It's interesting to note that even though the job didn't run a procedure
there were still a fair number of events that occured.  
Markup languages
generally need a head section at the top of the document which has
information about the document itself.  This section also holds
information about how the document should be formatted.  The very
last events are doc\_body and the end of doc.  Doc\_body is where
the output from our procedures will go.  

\index{Events!Initialize}
It is worth noting that the initialize event did not show up in the map.
We already know it is the first event that happens.  That is because the
short\_map tagset specifically omits it.  

The excercises in the next chapter are also very informative.  Particularly
the example which specifies all possible files on the ODS Markup statement
on page \pageref{allmaps}.

\begin{fvoutput}{event_map1_out}{Empty Event Map Output}
<doc operator="eric" sasversion="9.1" 
     saslongversion="9.01.01M0D10062003" 
     date="2003-10-24" time="18:09:16" encoding="iso-8859-1">
  <doc_head>
    <doc_meta/>
    <auth_oper/>
    <doc_title/>
    <stylesheet_link/>
    <javascript>
      <startup_function>
      </startup_function>
      <shutdown_function>
      </shutdown_function>
    </javascript>
  </doc_head>
  <doc_body>
  </doc_body>
</doc>
\end{fvoutput}

The output becomes much more interesting if the job is modified 
to run the print procedure with one observation.  As expected
there will now be table, row, header and data events.  But there
are also a lot of other new events.

\begin{fvcode}{event_map2}{Simple Event Map Output}
options obs=1;
ods tagsets.short_map file='test.xml';
proc print data=sashelp.class; run;
ods tagsets.short_map close;
\end{fvcode}

The output is rather extensive for such a small table!

\begin{sfvoutput}
<?xml version="1.0" encoding="iso-8859-1"?>

<doc operator="eric" sasversion="9.1" 
     saslongversion="9.01.01M0D10062003" 
     date="2003-10-24" time="18:27:50" encoding="iso-8859-1">
  <doc_head>
    <doc_meta/>
    <auth_oper/>
    <doc_title/>
    <stylesheet_link/>
    <javascript>
      <startup_function>
      </startup_function>
      <shutdown_function>
      </shutdown_function>
    </javascript>
  </doc_head>
  <doc_body>
    <proc name="Print">
      <anchor name="IDX"/>
      <page_setup>
        <system_title_setup_group>
          <title_setup_container>
            <title_setup_container_specs>
              <title_setup_container_spec/>
            </title_setup_container_specs>
            <title_setup_container_row>
              <system_title_setup value="The SAS System">
              </system_title_setup>
            </title_setup_container_row>
          </title_setup_container>
        </system_title_setup_group>
      </page_setup>
      <system_title_group>
        <title_container>
          <title_container_specs>
            <title_container_spec/>
          </title_container_specs>
          <title_container_row>
            <system_title value="The SAS System">
            </system_title>
          </title_container_row>
        </title_container>
      </system_title_group>
      <proc_branch name="Print" value="Print">
        <leaf name="Print" value="Data Set SASHELP.CLASS">
          <page_anchor/>
          <output name="Print">
            <table>
              <rowspec>
                <cellspec/>
                <cellspecsep/>
                <cellspec/>
                <cellspecsep/>
                <cellspec/>
                <cellspecsep/>
                <cellspec/>
                <cellspecsep/>
                <cellspec/>
                <cellspecsep/>
                <cellspec/>
              </rowspec>
              <colspecs>
                <colgroup>
                  <colspec_entry name="Obs"/>
                </colgroup>
                <colgroup>
                  <colspecsep/>
                  <colspec_entry name="Name"/>
                  <colspecsep/>
                  <colspec_entry name="Sex"/>
                  <colspecsep/>
                  <colspec_entry name="Age"/>
                  <colspecsep/>
                  <colspec_entry name="Height"/>
                  <colspecsep/>
                  <colspec_entry name="Weight"/>
                </colgroup>
              </colspecs>
              <table_headers>
                <header_spec>
                  <sub_header_colspec name="Obs">
                    <col_header_label value="Obs">
                    </col_header_label>
                  </sub_header_colspec>
                  <sub_header_colspec name="Name">
                    <col_header_label value="Name">
                    </col_header_label>
                  </sub_header_colspec>
                  <sub_header_colspec name="Sex">
                    <col_header_label value="Sex">
                    </col_header_label>
                  </sub_header_colspec>
                  <sub_header_colspec name="Age">
                    <col_header_label value="Age">
                    </col_header_label>
                  </sub_header_colspec>
                  <sub_header_colspec name="Height">
                    <col_header_label value="Height">
                    </col_header_label>
                  </sub_header_colspec>
                  <sub_header_colspec name="Weight">
                    <col_header_label value="Weight">
                    </col_header_label>
                  </sub_header_colspec>
                </header_spec>
              </table_headers>
              <table_head>
                <row>
                  <header value="Obs">
                  </header>
                  <header value="Name">
                  </header>
                  <header value="Sex">
                  </header>
                  <header value="Age">
                  </header>
                  <header value="Height">
                  </header>
                  <header value="Weight">
                  </header>
                </row>
              </table_head>
              <table_body>
                <row>
                  <header value=" 1">
                  </header>
                  <data value="Alfred">
                  </data>
                  <data value="M">
                  </data>
                  <data value="14">
                  </data>
                  <data value="69">
                  </data>
                  <data value="112.5">
                  </data>
                </row>
              </table_body>
            </table>
          </output>
        </leaf>
      </proc_branch>
    </proc>
  </doc_body>
</doc>
\end{sfvoutput}

That is a lot of output.  Obviously the probability of
using all of those events in one tagset is pretty slim.
But for every event there is probably a tagset that uses
it.  

\section{The Default\_Event Tagset Attribute}
\index{Attributes!Default\_event}
\index{Default\_event}
So how do you know that you are seeing all the events?  The
answer is because these tagsets use the default event.
Normally, if an event is not defined, the event request
is quietly forgotten.  But a tagset can have an event that
is designated as the default.  If an event is not found,
in a tagset that has a default event, the default event
is triggered instead.  The code for the shortmap tagset
can be seen 
in figure \ref{event_map3} on page \pageref{event_map3}.

The first thing is a bunch of
attributes at the top of the tagset.  
The attribute to look for is the default\_event.  In this 
case the default event is set to 'basic'.

\subsection{The Events}
\index{Events!Initialize}
The first event is the initialize event.  This empty definition ensures
that the initialize event request will be quietly dropped.

\index{show\_charset}
\index{suppress\_charset}
\index{SAS Registry!suppress\_charset}
The show\_charset event sets \$show\_charset variable if
suppress\_charset is not set.  Suppress\_charset is a setting
that can be made in the SAS registry.  This is the tagset's way
of recognizing it.


\index{Trigger}
\index{Statements!Trigger}
The doc event is defined almost the same as the basic 
event but prints some extra information that is good to
know. 

The doc event also does something we haven't seen yet.  It 
triggers the show\_charset event.  Trigger allows the creation
of arbitrary event definitions, that can be used as if they
are functions.
Triggers are extremely efficient.  At run time they behave as
if the triggered event's statement's were right where the trigger
statement is.  Similar to, but not quite the same as SAS macro's
or inline functions in lower level programming languages like 'C'.

\index{Putq}
\index{Statements!Putq}
The next new statement is putq.  Putq works just like put,
but it automatically double quotes any variables it prints.
This is perfect behavior for printing HTML or XML attributes.

\index{Variables!Event\_name}
After printing the the xml tag, The doc event creates the
doc tag by printing an < followed by the event\_name variable.
Event\_name will always have the name of the initial event
requested from within ODS.  In this case the value will be 'doc'.

\index{Value labels}
\index{Statements!Put!label/value pairs}
The rest of the start section is nothing but putq's.  
Looking at the doc tag in figure \ref{event_map1_out}
on page \pageref{event_map1_out} it is noticable that not all
of those putq's actually printed something.  This is because
The put statement always pairs variables with the any string that
immediately precedes it.  If the variable does not have a value,
then the string is not printed.  This is another behavior of put
that is perfect for printing HTML and XML attributes.

It's not so perfect behavior for debugging though.  It is sometimes
desirable to be sure that the put statement executed and that there
was nothing to print.  In that case, all that is needed is to separate
the label and the variable by inserting a shorter label inbetween.
This example separates the = from the label by making them into separate
strings.  The output will be the same, except 'Name' will always print
even when the name variable has no value.  Another technique is to separate
the string from the variable by using two put statements.

\begin{sfvcode}
    put 'Name' '=' name;

    put 'Name=';
    put name;
\end{sfvcode}

\subsection{The Default Event}
\index{Default Event}
The next event is the basic event.  The basic event won't be
found in any output from event\_map.  It's not one of the known 
events that ODS will request.  But it is the event assigned to
the default\_event attribute.  That means that the if the event
is not initialize or Doc then the 'basic' event will handle the
event request.

The basic event makes use of another variable 'empty'.  If empty
is true then this event does not encapsulate other
events.  In other words it's start and finish are triggered 
back to back with nothing in between.  This knowledge is used
to make the XML tag an empty\footnote{This is what a XML tag that contains 
no other tags is called} tag.

\index{Indent}
\index{ndent}
\index{Statements!ndent}
\index{xdent}
\index{Statements!xdent}
\index{Attributes!indent}
The value of empty is also used eliminate the ndent at the end of the start, and the entire 
finish section.  The ndent statement tells the tagset to increment it's
indention level by one.  The amount of space indented depends on the 
value in the indent attribute at the top of the tagset.  The xdent 
statement causes the indention level to decrease by one.

Finally, There is the last event attr\_out.   Attr\_out is triggered 
from the doc and basic events and is responsible for printing 
the name and label attributes within each tag.   This event is
mostly for convenience.  It makes it easy to add more attributes
to the map just by inheriting from this tagset and redefining this
Event.


\begin{fvcode}{event_map3}{Short Map Tagset}
    define tagset tagsets.short_map;
        notes "This is the event map definition";
        map = '<>&"';
        mapsub = '/&lt;/&gt;/&amp;/&quot;';
        nobreakspace = ' ';
        split = ' ';
        indent=2;
        stacked_columns = yes;
        output_type = 'xml';

        default_event = 'basic';

        define event initialize;
        end;

        define event show_charset;
           set $show_charset "1" / if !exists(suppress_charset);
        end;

        define event doc;
            start:
                trigger show_charset;
                put '<?xml version="1.0"';
                putq " encoding=" encoding / if exists($show_charset);
                put "?>" CR CR;
                put "<" EVENT_NAME ;
                putq " title="        BODY_TITLE;
                putq " author="       AUTHOR;
                putq " operator="     OPERATOR;
                putq " sasversion="   SASVERSION;
                putq " saslongversion="   SASLONGVERSION;
                putq " date="         DATE;
                putq " time="         TIME;
                putq " encoding="     ENCODING / if exists(\$show_charset);
                putq " language="     LANGUAGE;
                putq " trantab="      TRANTAB;
                trigger attr_out;
                put ">" CR;
                ndent;
            finish:
                xdent;
                put "</" EVENT_NAME '>';
        end;

        define event basic;
            start:
                put "<" EVENT_NAME ;
                trigger attr_out;
                put "/" / if exists(empty);
                put ">" CR;
                break / if exists(empty);
                ndent;
            finish:
                break / if exists(empty);
                xdent;
                put "</" EVENT_NAME ">" CR;
        end;

       define event attr_out;
          putq " name=" name;
          putq " value=" value;
       end;
    end;

\end{fvcode}


\subsection{summary}
The mapping tagsets are the best resource for 
finding events are not known.  They reveal 
all the events for any given SAS job.  
There are lots of ways to find events. 
One of the easiest ways is to examine the existing
 tagsets and their output.
The mapping tagsets are another resource that can
be extremely helpful.  Adding a default event
to an existing tagset can also be extremely helpful
for revealing new events between the events already
defined.

\section{Finding Variables}
\index{Variables!Finding}
\index{Finding Variables}
In addition to knowing when events occur it's also
important to know what data is available for any 
given event.  The mapping tagsets show some of this
but there are far too many variables to print them
all the time.

\index{Statements!Putvars}
\index{Putvars}
There is one tagset that prints all the variables
for all the events.  The Odsxrpcs tagset uses a
statement called putvars to do this.

\index{Tagsets!Odsxrpcs}
\index{Odsxrpcs}
To keep our example short this tagset defines only one event.
The table event is familiar so that is a good place to start.
The ODSxrpcs tagset uses a similar event for it's default event.

The entire tagset is not much more than a single putvars statement. 
See figure \ref{putvars1}.

\begin{fvcode}{putvars1}{Printing Event Variables}
proc template;

    define tagset tagsets.my_map;

        define event table;
            putvars event _name_ ' : ' _value_ nl;
        end;

     end;
run;

ods tagsets.my_map file='putvars.txt';
proc print data=sashelp.class; run;
ods tagsets.my_map close;
\end{fvcode}

The output in figure \ref{putvars1_out} shows all the populated 
event variables for the print procedure's table event.  

\begin{poutput}{putvars1_out}{Table Event Variables Output}
type : table
anchor : IDX
event_name : table
encoding : iso-8859-1
operator : eric
date : 2003-10-25
sasversion : 9.1
saslongversion : 9.01.01M0D10062003
time : 01:00:07
state : start
proc_count : 1
total_proc_count : 1
page_count : 1
total_page_count : 1
dest_file : body
bodyname : putvars.txt
tagset : TAGSETS.MY_MAP
style : Default
javadate : 2003-10-25
javatime : 01:00:07-04:00
data_viewer : Report
style_element : Table
\end{poutput}

\index{Event!Put\_all}
\index{Statements!Putvars!Put\_all event}
Adding the other variable category's and some indentation makes
the output more informative and readable.
The better version of the tagset.
is shown in figure \ref{putvars2} on page \pageref{putvars2}.
The new, nicely formatted, output is shown in 
figure \ref{putvars2} on page \pageref{putvars2}.

\begin{fvcode}{putvars2}{Printing all Variables}

proc template;

    define tagset tagsets.my_map;

        indent=2;

        define event table;
            trigger put_all;
        end;

        define event put_all;
            put 'Event: ' event_name nl;
            ndent;

            put 'Event:' nl;
            ndent;
            putvars event _name_ ' : ' _value_ nl;
            xdent;

            put 'Style:' nl;
            ndent;
            putvars style _name_ ' : ' _value_ nl;
            xdent;

            put 'Memory:' nl;
            ndent;
            putvars memory _name_ ' : ' _value_ nl;
            xdent;

            put 'Stream:' nl;
            ndent;
            putvars stream _name_ ' : ' _value_ nl;
            xdent;

            put 'Dynamic:' nl;
            ndent;
            putvars dynamic _name_ ' : ' _value_ nl;
            xdent;

            xdent;
            put nl;
        end;

     end;
run;

ods tagsets.my_map file='putvars.txt';
proc print data=sashelp.class; run;
ods tagsets.my_map close;

\end{fvcode}

\begin{poutput}{putvars2_out}{Table Event Variables Output}
Event: table
  Event:
    type : table
    anchor : IDX
    event_name : table
    encoding : iso-8859-1
    operator : eric
    date : 2003-10-25
    sasversion : 9.1
    saslongversion : 9.01.01M0D10062003
    time : 01:13:14
    state : start
    proc_count : 1
    total_proc_count : 1
    page_count : 1
    total_page_count : 1
    dest_file : body
    bodyname : putvars.txt
    tagset : TAGSETS.MY_MAP
    style : Default
    javadate : 2003-10-25
    javatime : 01:13:14-04:00
    data_viewer : Report
    style_element : Table
  Style:
    FRAME : BOX
    RULES : GROUPS
    HTMLCLASS : Table
    CELLSPACING : 1
    CELLPADDING : 7
    FRAMEBORDER : auto
    CONTENTSCROLLBAR : auto
    BODYSCROLLBAR : auto
  Memory:
  Stream:
  Dynamic:
\end{poutput}

\section{The Putlog Statement}
\index{Putlog}
\index{Statements!Putlog}
This chapter has shown how to reveal the inner workings 
of tagsets.  There
is another statement that is quite handy for quickly showing
what is going on.
Putlog is like put except that instead of
printing to the output file it prints to the SAS log.  

Putlog is nice for checking values or the order of events
without putting debug messages in the output file.  
It also works nicely for printing usage messages from the tagset.

\section{Define, Identify, Locate, Explore, and Solve.}
This section will outline a method to solving problems with tagsets.
The key to any solution is knowing what events are in the places that
can help with the solution, and what information those events have.    

\begin{enumerate}
\item {\bfseries Define:}
\begin{enumerate}
\item {\bfseries Understand the problem.}
The first step is to understand what the desired output should look
like.  Modifying ODS output to the desired format is the easiest way
to verify the goal.
\end{enumerate}

\item {\bfseries Identify:}
Finding the events that need modification or definition is the 
first step towards solving any problem with a tagset.  
\begin{enumerate}
\item Find the Markup output that needs changing.
\item Identify a unique string or data value
\item {\bfseries Identify the event:}
\item {\bfseries Search the Source.}
If the string is not data driven, it is sometimes reasonable to
find the string in the original tagset.
\item {\bfseries Search the Mapping Output.}
Run the SAS job using a mapping tagset and search for a known string.
This is particularly useful if the SAS job provides a unique string to 
search for.
\end{enumerate}

\item {\bfseries Locate the event:}
\begin{enumerate}
\item locate the event definition in the source.
Find the event that was identified in the mapping tagset's output in the parent tagset.
\end{enumerate}

\item {\bfseries Explore:}
\begin{enumerate}
\item {\bfseries Copy the Parent's definition.}
Copy and paste the parent tagset's event definition into the new 
tagset definition. If the identified event is not defined by the parent, 
then define it. 

\item {\bfseries Identify Resources.}
The next step is to figure out what to change in the event.  Sometimes
the changes are obvious.  Regardless, it is always helpful to know 
what variables are bundled in the event request and what their values 
are.  

\begin{enumerate}
\item {\bfseries Identify Variables.}
Adding a putvars statement or triggering a put\_all type event
is the next step.  
Run the SAS job using the new tagset, and look at the output file for 
the new event.  Now there is a list of values that should enable a
determination of the final event definition.
\end{enumerate}
\end{enumerate}


\item {\bfseries Problems}
\begin{enumerate}
\item {\bfseries Use putlog.}
If there are problems it is sometimes useful to add a putlog statement to
see if things are going right.  More putvars statements for more variable
categories is also helpful.

\item {\bfseries Add a default event.}
It is sometimes useful to add a default event that prints the event name.
Any undefined events will then show up between the events already defined.
If an event looks useful, add a definition with a putvars statement to see
what sort of data the event contains. 
\end{enumerate}

\item {\bfseries Solve:}
\begin{enumerate}
\item {\bfseries Final version.}
Remove any debugging statements and make the final adjustments.
\end{enumerate}
\end{enumerate}

\subsection{Repeat as Necessary}
Most of the work happens in the identify, locate, and explore phase of
the process.  Usually some new tagset code results from knowledge provided
by those three steps.  But that is not usually the final solution.
It is quite common to repeat these steps several times
in the course of writing a new tagset.

\section{Going step by step}

Many of the problems that tagsets solve are fairly simple and involve a few
events at most.  
This next example is representative of most common type of tagset modification.
The key to solving this sort of problem is not so much the coding, but finding
the place to put the code. The example will show how the steps of 
Define, Identify, Locate, explore and solve, are used.

\subsection{Adding a Target to a URL}
The Report procedure has a very handy option that allows dynamic creation
of URL's based upon data.  This feature works great for creating drill down
reports that load successively detailed tables.  But if the desire is for 
the page to load in a new window this approach will not work. This is because the 
target attribute will not carry through no matter how it is set.  This can 
be fixed with a tagset.

\subsection{Define the Problem}
The following code shows Report procedure code that dynamically assigns
a url based upon the data in each observation.  Despite various efforts
to give the url a target attribute of '\_blank\_' all efforts have failed.
The result is that clicking on a link replaces the current page with the
new one.  The desire is to get another window with the new page.
Looking forward to the next step, this job has already added an additional ods
statement, event\_map.

\begin{sfvcode}
data x; x=1; x2=1; x3=1; x4=1; run;

ods html file="report_url.html";
ods tagsets.event_map file="report_url.xml";

proc report data=x nowd;
  column x x2 x3 x4;
  define x / display;
  define x2 / display;

  compute x2;
    call define( _COL_, "URL", "URL-from-REPORT" );
    endcomp;

  run;

ods _all_ close;
\end{sfvcode}

This is just a simple, do nothing, example.  But it gives us an HTML table
cell that looks like this.

\begin{sfvoutput} 
<td class="r Data"><a href="URL-from-REPORT">        1</a>
\end{sfvoutput} 

There are various methods of setting the target style attribute on a column.
However, none of them will work with a dynamic url. Our goal is a cell that
looks like this.

\begin{sfvoutput} 
<td class="r Data"><a href="URL-from-REPORT" target="_blank_">        1</a>
\end{sfvoutput} 

\subsection{Identify the event}
\index{Events!Identify}
\index{Events!Finding}
\index{Tagsets!Event\_map}
Looking in the html tagset source for 'href' reveals several events that could
be the one.  A little intuition might allow an accurate guess, 
but using the event\_map tagset will remove all doubt.
Adding the event\_map ods statement to the job is all that is needed.
Looking for - 'URL-from-REPORT' reveals these events.

\begin{sfvoutput} 
<data event_name="data" trigger_name="attr_out" output_name="Report" 
      output_label="Detailed and/or summarized report" 
      section="body" class="Data" row="2" data_row="1" 
      colcount="4" col="2" type="string" index="IDX" just="r">

  <hyperlink event_name="hyperlink" trigger_name="attr_out" 
              output_name="Report" 
              output_label="Detailed and/or summarized report" 
              value="        1" colcount="4" index="IDX" 
              just="c" url="URL-from-REPORT"/>
</data>
\end{sfvoutput} 

\index{Event!Hyperlink}
This is quite verbose but it tells exactly which event is needed, the hyperlink
event.  Finding events is an important skill
when it comes to working with tagsets.  This method is one of the primary
ways to do that.

\subsection{Locate the Event}
\index{htmltags.tpl}
\index{Tagsets!HTML4}
\index{Tagsets!HTMLcss}
\index{Tagsets!PHTML}
Now that the event is identified the next step is to find it
in the html tagset.  Technically ODS HTML uses the html4 tagset, but that
tagset inherits most of it's events from the htmlcss and phtml tagsets.
Thankfully all of those tagsets come bundled in one file, htmltags.tpl.
This is another disadvantage of using the template procedure's source
statement.  It would take three source statements to see all of the events
in the html4 tagset.  The htmltags.tpl source file has everything in one place.

\begin{sfvcode}
   proc template;
   source tagstes.html4;
   source tagstes.htmlcss;
   source tagstes.phtml;
   run;
\end{sfvcode}

\index{Event!Hyperlink}
\index{Tagsets!HTML4}
\index{Tagsets!HTMLcss}
\index{Tagsets!PHTML}
\index{Tagsets!MVSHTML}
Searching the file finds three hyperlink events.  One in phtml,
a slightly more complex definition in htmlcss and a completely different definition
in mvs\_html.  When working with the html4 or htmlcss tagsets, the event definitions
are always more complex than their phtml counterparts.
The event defined in htmlcss is the one that is needed since that is the one that html4 will
use.  It looks like this.

\begin{sfvcode}
        define event hyperlink;
          start:
            put '<a href="' URL;
            /*put "#" ANCHOR;*/
            put '"';
            putq " target=" HREFTARGET;
            put ">";
            trigger do_value;
          finish:
            put "</a>" CR;
        end;
\end{sfvcode}

\subsection{Explore the Data}
Adding a putvars statement or triggering put\_all doesn't really help much.  What
it shows is that even if hreftarget is set in the style, The report procedure
removes it.  

The target line in the hyperlink event needs to have an alternate behavior.  Causing
the alternate behavior by indicating a table column will do.
Macro variables are the easiest way to do
that.  One macro variable can tell which column needs the target, and another
can give the target value.   

\begin{sfvcode}
    %let target_value=_blank_;
\end{sfvcode}
  

Adding this to the hyperlink event should get us close.  But it will have
to be conditional.  So this isn't a complete answer.
\begin{sfvcode}
        putq " target=" target_value;
\end{sfvcode}

\subsection{Repeat.  Identify, Locate, Explore} 
The hard part is over.  The event has been identified, located and the data
it contains is known.  However, this is only a partial solution.  The final
solution needs to preserve the current behavior while adding a new feature 
that is driven by macro varibles.

But there is still one problem. The hyperlink event doesn't know anything about
table columns.  That will make it difficult to cause the alternate behavior.

Looking at the original event\_map output it can be seen that the data event knows
the column number.  Colstart was used earlier in the CSV examples.  It can be used
again here.  But it will have to be saved away in a variable.  The next step
is to copy and paste the data event from the htmlcss tagset.  The only changes are
to set \$column to colstart and to unset it at the end.  Now the hyperlink event
can use the \$column variable to check the column number against the new macro 
variable.

\subsection{The solution}

\index{Variables!target\_column}
\index{Variables!target\_value}
Looking at the new hyperlink event and the new target\_column and target\_value
variables it should be fairly obvious that these variables tell which column to
add a target to, and what that target should be.

\begin{sfvcode}
    define tagset tagsets.targethtml;
        parent=tagsets.html4;
    
        mvar target_column;
        mvar target_value;
    
        define event hyperlink;
            start:
                put '<a href="' URL;
                /*put "#" ANCHOR;*/
                put '"';

                do /if hreftarget;
                    putq " target=" HREFTARGET;
                else /if cmp($column, target_column);
                    putq " target=" target_value /if target_value;
                done;
            
                put ">";
                trigger do_value;
            finish:
                put "</a>" CR;
        end;
    end;
\end{sfvcode}

\index{Event!data}
\index{Variables!colstart}
\index{Statements!Set}
\index{tagsets!HTMLcss}
\index{Procedure!Report}
This tagset will work perfectly most of the time.  The exception is the report procedure.
The report procedure frequently
gives a data event with no data.  Then does something else between the start and
finish of the data event.  The only real problem with that is the metadata and the 
context that metadata provides.  In this case column number of the cell was needed.
The set statement is used to set \$column to colstart.  When the
hyperlink event comes along, it has something to check against the macro variable.
The good news is that the data event, with the exception of the set \$column statement,
is an exact copy of the data event from the htmlcss tagset.

Running the following proc print with the event\_map tagset will shows a different
data event than the one provided by report.  Ultimately the data event triggers the
hyperlink event, so our solution still works.  But for print, it would have been
unessecary to save away the column number.

\begin{sfvcode}
proc print data=x;
    var x;
    var x2 / style(data) = [url="URL-from-Print"];
    var x3;
    var x4;
run;
\end{sfvcode}

The XML created for a data cell looks like this.

\begin{sfvoutput}
   <data event_name="data" trigger_name="attr_out" output_name="Print" 
     output_label="Data Set WORK.X" section="body" class="Data" value="1" 
     row="2" data_row="1" colcount="1" col="3" type="double" 
     rawvalue="P/AAAAAAAAA=" index="IDX1" just="r" url="URL-from-Print">
   </data>
\end{sfvoutput}

\index{Procedure!report}
Realizing that this needed to be done for the report procedure only comes with experience. 
This problem will show up several times in this book.  At some point, realizing
that the report procedure is special will become second nature.  This problem will
go away in the future, as improvements are made to proc report. 
The final solution is shown below.

\begin{fvcode}{Target_html}{The complete Target HTML tagset}
proc template;
    define tagset tagsets.targethtml;
        parent=tagsets.html4;
    
        mvar target_column;
        mvar target_value;
    
        define event hyperlink;
            start:
                put '<a href="' URL;
                /*put "#" ANCHOR;*/
                put '"';

                do /if hreftarget;
                    putq " target=" HREFTARGET;
                else /if cmp($column, target_column);
                    putq " target=" target_value /if target_value;
                done;
            
                put ">";
                trigger do_value;
            finish:
                put "</a>" CR;
        end;

        define event data;
            start:
                trigger header /breakif cmp(htmlclass, "RowHeader");
                trigger header /breakif cmp(htmlclass, "Header");

                /* The only new line here */
                set $column colstart;

                put "<td";
                putq " title=" flyover;
                do /if !cmp(htmlclass,'batch');
                    trigger classalign;
                    trigger style_inline;
                done;
                trigger rowcol;
                put " nowrap" /if no_wrap;
                put ">";
                trigger cell_value;
            finish:
                trigger header /breakif cmp(htmlclass, "RowHeader");
                trigger header /breakif cmp(htmlclass, "Header");

                trigger cell_value;
                put "</td>" CR;

                /* The only new line here - unset just to be safe */
                unset $column;
            
            end;
    
    end;
run;


data x; x=1; x2=1; x3=1; x4=1; run;

%let target_column=2;
%let target_value=_blank_;

ods tagsets.targethtml file="report_url.html";

proc report data=x nowd;
  column x x2 x3 x4;
  define x / display;
  define x2 / display;

  compute x2;
    call define( _COL_, "URL", "URL-from-REPORT" );
    endcomp;

  run;

ods _all_ close;
\end{fvcode}

The output gives the desired reselt.  The same cell in the output now looks 
like this.

\begin{sfvoutput}
<td class="r Data"><a href="URL-from-REPORT" target=_blank_>        1</a>
\end{sfvoutput}

\index{Statements!Putq!Macro variable bug}
Notice that putq didn't do it's job.  This appears to be a bug
with macro variables in SAS 9.1.  The best solution for that is to put the
quotes around the variable yourself.  Using a put instead of a
putq will insure that the tagset won't break when a future release
fixes the quoting bug.   

\begin{sfvcode}
                do /if hreftarget;
                    putq " target=" HREFTARGET;
                else /if cmp(\$column, target_column);
                    put ' target="' target_value '"' /if target_value;
                done;
\end{sfvcode}

Another method is to assign the macro variable
to another variable with a set.  Then use that variable in the putq.

\begin{sfvcode}
                do /if hreftarget;
                    putq " target=" HREFTARGET;
                else /if cmp($column, target_column);
                    do /if target_value;
                        set $target_value target_value;
                        putq ' target=' $target_value;
                        unset $target_value;
                    done;
                done;
\end{sfvcode}


\section{Summary}
Tagsets provide some powerful tools which can help find the 
best way to write or change any tagset.  When starting from
scratch one of the mapping tagsets is a good place to start.
They are also good for comparing against existing tagsets.

The default event can be used in an existing tagset to reveal
unknown events between the events already defined.

The putvars statement can be used to show all the currently populated
variables.  Including a put\_all event in a tagset gives a quick
and easy way to find that elusive variable when needed.

The putlog statement can be used to print information to the log which
can make it easier to find or see variables or sequences of events. They
are particularly useful as a sanity check, when it's not clear that a
tagset is doing what it should.



