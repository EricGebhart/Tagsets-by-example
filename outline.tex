\documentclass{book}
\usepackage{fancyhdr, flafter, makeidx, float, times}
\usepackage[T1]{fontenc}
\usepackage{graphicx}
\usepackage{fancyvrb}
\usepackage[color]{ods}
\pagestyle{fancy}
\makeindex
%includeonly{Introduction, chap1}
\begin{document}
\fancyhead{} %clear all fields
\fancyhead[LO,RE]{\thepage}
\fancyhead[RO]{}
\fancyhead[LE]{}
\fancyfoot[CO,CE]{}
\frontmatter
\title{ODS Markup: \
Tagsets by Example}
\renewcommand{\today}{September 22, 2003}
\pagenumbering{roman}
\maketitle
  \section*{Preface}

The SAS Output Delivery System's (ODS) markup destination appears
to be much like all the other destination types that ODS is capable
of.  In reality it is quite different.  

The other ODS destinations have a
fixed output type.  The RTF destination can only create RTF output.
The printer destination can create postscript, PCL, and PDF.  But ODS
Markup can create any number of output types, all of which can be
created or modified by anyone.

ODS Markup is able to do this because it is really a framework which
uses a tagset to define what it should print to it's files.  The other
destinations, including the old HTML destination were hard coded
and compiled as a part of the SAS executable.  

A tagset can be described as a group of events or functions which use
a programming syntax similar to other programming languages such as
perl, python, shell, and datastep. 

We can create and use as many different tagsets as we like.  
There are a number of tagsets that come standard as a part of SAS.

Because there are so many different tagsets, ODS Markup is rather 
chameleon-like.  ODS HTML is really ODS Markup.  So is the CSV 
destination, and CHTML, excelXP, LaTeX, Troff, etc.  The only 
difference between these different ODS destinations is the tagset 
that is in use.  They are all really ODS Markup, we just don't call
it that.
 
When we use ODS Markup we are using a tagset.  The tagset determines
the type of output. Therefore, a tagset defines an output destination!
This realization has far reaching implications.  It means that we
can change the html that ODS HTML generates.  It means we can create
a new HTML destination with your corporate style and headings.  We
can create a new XML destination to allow data interchange with a
business partner or client. We can even create a simple flat file
format that can simplify processing by other programs. The output
can be modified and adapted in ways that could only be imagined until now.
The possiblities are endless.

It stands to reason, that if we want to use ODS Markup to our best
advantage, what we really want to do is use tagsets.  That
is what this book is going to explore.  What are these tagsets and how
do we use them?  You will find that tagsets can simplify otherwise
complex problems in a way that allows reuse and flexibility that
would not be possible with out them.

\cleardoublepage
	ableofcontents
\cleardoublepage
\mainmatter
\cleardoublepage
\pagenumbering{arabic}
\setlength{\headrulewidth}{1pt}
\lhead[\thepage]{\sl\rightmark}
\rhead[\sl\leftmark]{\thepage}
\part{Introduction}
\chapter{Introduction}
ODS markup and tagsets are the best way to create markup output
from SAS.  This chapter will talk about some of the reasons for
learning tagsets.  As well as what they are and the history 
behind them.
\section{Why Learn Tagsets?}

\section{What are Tagsets?}

\section{A short History of Tagsets}

\section{Markup Languages}

\subsection{In the Beginning There Was Roff}

\subsection{Then there was LaTeX}

\subsection{SGML and friends}

\subsection{Other Formats}

\subsection{What it all means}

\chapter{The basics}
This chapter will cover the basics of using ODS Markup and the Template Procedure.
ODS markup relies upon tagsets to define how it should work.  
Indicating which tagset to use is the only thing that
differentiates the ODS Markup statement from any other ODS destination.
A tagset is defined using the template procedure.  The amount of
template knowledge needed is not that great, but it is good to know
the basics of using it.
\section{Tagsets and Proc Template}

\subsection{The tagset directory}

\section{Using a tagset}

\section{Summary}

\chapter{Tagsets: how do they work?}
Tagsets are a hybrid of procedural and non-procedural, declaritive
programming languages.  A tagset is a collection of event definitions.
Each definition can have procedural elements, but the events themselves
are not procedural.  That's why they are called events.  Events do happen
in some order, but with variation.  They are like the events in your
day.  The alarm clock, brushing your teeth, eating breakfast, going
to work, etc.  They do have some predictability, but are not exactly
the same all the time.  This chapter will use a simple plain text
tagset to show how tagsets work.  As the tagset evolves the 
output created will reveal the fundamentals of tagset behavior. 
\section{Data and context in time}

\section{Event Requests}

\subsection{A few variables}

\section{Our First tagset}

\subsection{Define statement}

\subsection{The data event}

\subsection{The header event}

\section{Fleshing out the plain\_text tagset}

\subsection{head, body, and foot}

\subsection{Titles}

\section{Summary}

\chapter{Modifying Existing Tagsets}
The most common tagset programming is done to
modify the behavior of an existing tagset. 
Modifying a tagset is fairly easy to do because
tagsets can inherit events from each other.  
It is also easy to identify which events need changing.
This chapter will introduce a few new concepts
while showing the steps needed to modify a tagset
through inheritance.  
\section{Changing the delimiter for CSV files}

\subsection{Finding the Events}

\subsection{Simple If's}

\section{Making the changes}

\section{A better CSV tagset}

\subsection{Macro variables}

\subsection{The Initialize Event}

\subsection{The set statement}

\subsection{The New CSV Tagset}        

\subsection{Tagset Alias}

\section{Summary}

\chapter{The Path to Enlightenment}
So far all of the examples have given away the events and
variables needed.  But the ability to find events, and their
variables is required to be truly effective at programming
tagsets.  This chapter will explain several techniques that
will illustrate which events
may work for a solution, and the variables that are bundled in
those events.  In short, this chapter will show how to get 
tagsets to reveal themselves.
\section{Finding Events}

\subsection{The Short\_map Tagset}

\section{The Default\_Event Tagset Attribute}

\subsection{The Events}

\subsection{The Default Event}

\subsection{summary}

\section{Finding Variables}

\section{The Putlog Statement}

\section{Define, Identify, Locate, Explore, and Solve.}

\subsection{Repeat as Necessary}

\section{Going step by step}

\subsection{Adding a Target to a URL}

\subsection{Define the Problem}

\subsection{Identify the event}

\subsection{Locate the Event}

\subsection{Explore the Data}

\subsection{Repeat.  Identify, Locate, Explore} 

\subsection{The solution}

\section{Summary}

\chapter{File Redirection}
Adding output to the all the files specified on the ODS 
statement requires that some events be redirected to those
files.  This chapter will show the differences between the
various files and the events associated with them.  It will
also show how to use file redirection modify their contents.
\section{The File Attribute}

\section{Identify}

\section{Locate and Explore}

\section{File interactions}

\section{Freedom of Choice}

\subsection{Explore}

\section{Summary}

\part{Technicalities}
\chapter{The Tagset Attributes}
A tagset can have several attributes.  The attributes control the basic
behavior of the tagset.  Most importantly the translation of special
characters that may interfere with the markup language being generated.
Other attributes control output indention, breaking of lines, even the
way the output should be displayed in the SAS output window.  Setting 
up a tagset's attributes is the first step towards creating a new tagset.
\section{Parent}

\section{Special Characters}

\subsection{Automatic Character Translation}

\section{Non Breaking Spaces}

\section{Split Characters}

\section{Indention}

\section{Stacked Columns}

\section{Image Formats}

\section{Output Type}

\section{Adding Measurements}

\section{Copyright Symbol}

\section{Trademark Symbol}

\section{Registered Trademark Symbol}

\section{Default Event}

\section{Embedded Stylesheet}

\section{Pure Style}

\section{lognote}

\section{Splitting Text}

\subsection{Breaktext Length}

\subsection{Breaktext Width}

\subsection{Breaktext Ratio}

\section{External Graph Instance}

\section{No Byte Order Mark}

\section{Hierarchical Data}

\section{Summary}

\chapter{Creating Variables}
The set statement has already been used in several examples. 
Set is the primary way to create and modify variables in
tagsets.
Before continuing it would be good to
explain the different types of variables and the methods 
of creating them.  This chapter is dedicated to explaining
the different types of variables that can be created and 
used in tagsets.
\section{String Variables}

\section{Lists}

\section{Dictionaries}

\section{Numeric Variables}

\section{Stream Variables}

\subsection{Stream Specific Statements}

\section{The Putvars Statment}

\section{Bringing it all together}

\section{Summary}

\chapter{Procedural controls}
The tagset's if syntax has already been used in previous examples. 
With SAS 8.2 and 9.0 if statements were kept to a few simple 
choices.  This simple if statement is still the most efficient
and most frequently used type of conditional programming that
tagsets have.  With SAS 9.1 if statments have been extended with
where clauses and new statements to provide fully featured procedural
controls.  
The examples in this chapter will explain the if syntax and the procedural control
statements that complement them.
\section{Simple If's}

\subsection{Built in tests}

\section{Where Clauses}

\section{Break}

\section{Breakif}

\section{Do blocks}

\section{Do While Loops}

\section{Iterating through Dictionaries}

\section{Bringing it all together}

\section{Summary}

\chapter{Trigger Happy}
ODS triggers, or requests, many events internally.  
It is also possible to trigger
events from other events. This chapter is going to
explain the trigger statement and how to use it.  
\section{Simple Triggers}

\section{Events with a state}

\section{summary}

\chapter{Data Step Functions}
Data step functions provide important capabilities to tagsets.  Many
solutions are dependent on string manipulation, formatting and 
number conversion.  Without the functions, these solutions would
be very difficult or impossible.  This chapter will explain the usage
of data step functions in tagsets.
\section{Set and Put statements}

\section{The Eval Statement}

\subsection{number conversions}

\section{advanced usage and debugging}

\subsection{File I/O}

\subsection{Perl Regular Expressions}

\section{Summary}

\part{Intermediate Examples}
\chapter{Styles and Tagsets, A perfect match}
Understanding how to use ODS styles with tagsets can
simplify many problems.  Styles can help make
a tagset more versatile and reusable.  They can
also help to make the output less verbose and
more compartmentalized. This chapter 
will demonstrate the basics of using styles 
with tagsets.   The best way to start is with
a simple tagset designed specifically
to show how styles work.  
\section{Starting Simple}

\subsection{Embedded styles}

\section{Styles of your own choosing}       

\section{Getting the whole style}       

\section{Summary}

\chapter{Tagsets with Style}
The fundamentals of tagsets have been covered, now it is time
do something with them. 
This chapter will cover examples that specifically leverage
ODS styles to accomplish their tasks.
\section{A Problem with Table Rules}

\subsection{Define the problem}

\subsection{A Simple Solution}

\subsection{Identify and locate the event}

\subsection{A Simple Solution}

\subsection{Table Rules with style}

\subsection{The Better Solution}

\section{Everyone likes stripes}

\subsection{Defining the Problem}

\subsection{The HTML solution}

\subsection{The Code}

\subsection{The LaTeX solution}

\subsection{The Code}

\subsection{Summary}

\section{slidebar columns}

\subsection{Define the Problem}

\subsection{Identify and Locate}

\subsection{The Solution}

\subsection{Example Summary}

\section{Summary}

\chapter{Tagsets with Streams}
Streams add another level of versatility to tagsets.  
Streams allow for output to be saved away, and reused or
just delayed.  This is ideal for solving problems where
the ODS event model does not match the shape of the target
ouput. Streams are the most common solution to this problem.
Streams can delay output, collect output from different events
at different times, to be used later.
This chapter will show how to use streams to solve problems
that would otherwise be impossible.
\section{Overly long Tables}

\subsection{Defining the Solution}

\subsection{Identify, Locate and Explore}

\subsection{The Solution}

\subsection{summary}       

\section{Special Cases} 

\section{One Better}

\subsection{The solution}

\section{A Tagset with Startpage}

\subsection{Identify, Locate and Explore}

\subsection{Defining the solution}

\subsection{Block and Unblock statments}

\subsection{A partial solution}

\subsection{The Solution}

\subsection{More Identify and Locate}

\subsection{Refining the Control}

\subsection{Almost There}

\subsection{The Final Solution}

\section{Summary}

\part{Advanced Examples}
\chapter{ODS Output for Website Integration}
Many web sites have a framework that web pages must fit into 
before they will be published.  Usually this means
editing the html pages and pasting in some chunk of HTML from
some files the website authors have provided.
Usually there is one file to replace the top section of the
web page and another to replace the bottom.  
The tagsets in this chapter will create HTML that
is ready for integration with any website.  
\section{The Problem}

\section{Alternate Behavior for existing options}

\subsection{Explore}

\section{Reading an external file}

\section{The Solution}

\subsection{Initialization Timing}

\section{Summary}

\chapter{Datastep Conversions}
You may wonder, why would someone want to convert a datastep 
program to use tagsets?  There are lots of reasons.  Mostly it's
for code reuse and flexibility.  Datastep programs allow lots of
flexibility but they are written for one purpose.  If the same 
functionality is desired for a different set of data an entirely
new datastep will have to be written.  If the rendering is left to
a tagset, then anyone can use that tagset with any form of data,
and any procedure.  Maintenance of the code is also simplified 
since both the tagset and the resulting SAS job wil be much
simpler than the original datastep. Add the various powers of
ODS to this new found flexibility and there is plenty of motivation
for conversion.  This chapter will show the process and rewards 
of converting datastep programs to tagsets. 
\section{Special Bylines}

\subsection{The DataStep Code}

\subsection{Breaking it down}

\subsection{The Style}

\subsection{Counting Observations}

\subsection{various problems}

\subsection{Modifying the Byline}

\subsection{A more flexible solution}

\section{Slidebars for HTML, PDF, and PS}

\subsection{breaking it down}

\subsection{The Style}

\subsection{The HTML Tagset}

\subsection{Dealing with the Report Procedure}

\subsection{The LaTeX Tagset}

\section{Summary}

\chapter{Extended examples}
This chapter will explore some more complex examples.  The
complexity comes mostly from the combination of multiple
techniques.  From a juggler's point
of view, there are just more balls in the air.  But a ball is
still just a ball.  Any given feature of these new tagsets
is still just as simple as it ever was.  This is one of the
wonderful things about tagsets.  It is easy to reuse and 
combine ideas from different tagsets.
\section{Repeating Headers, and Mirrored Row headers}

\subsection{The Single Stream Solution}

\subsection{The Multiple Stream Solution}

\subsection{The Multiple List Solution}

\section{Automatic Panelling}

\subsection{An Extension of Start Page}

\subsection{The solution}

\section{HTML forms}

\subsection{Option Lists}

\subsection{Saving the Lists}

\subsection{Creating the Form}

\section{Summary}

\chapter{A feature Rich Tagset}
In this book we have created many tagsets that can each do something special.
The tagsets have been carefully written to enable those features to be
transparent to anyone who uses them.  It is possible that all of these features
be combined into one tagset.  So that is what we are going to do.
\section{Which features?}

\section{Lining up the inheritance}

\section{Copy and Paste}

\section{Macro Variables and Tagset Alias}

\section{Summary}

\part{Usage Notes and Caveat's}
\chapter{Special Cases, Procedures and Operating Systems'}
The examples in this book have shown that there are some problems
when it comes to the Report and Tabulate procedures.  These
problems are surmountable but they do cause some pain.  This
chapter is going to explain the specifics of these problems
and how to work around them.
\section{A Report Procedure problem}

\subsection{deferred data}

\section{The Tabulate and Report problem}

\subsection{The Table Head section}

\subsection{The Table column specifications}

\section{HTML on MVS with a PDSE}

\section{Summary}

\chapter{Using LaTeX}
Besides HTML and XML, LaTeX output is one of the most useful
and versatile output destinations available.  LaTeX is commonly
used in publishing and is capable of creating several viewable
formats, including PDF and Postscript.  LaTeX supports color
and various image formats.
Sadly, this destination is largely overlooked.
This chapter will discuss how to use it to advantage. 
\section{The LaTeX statement}

\subsection{Color Support}

\section{Compiling the LaTeX Output}

\subsection{The latex Command}

\subsection{The dvips Command}

\subsection{The pdflatex Command}

\section{Integrating LaTeX output into documents}

\subsection{The easy way}

\subsection{Using NewFile to advantage}

\subsection{The simple way}

\section{Image Formats and Graph}

\section{LaTeX in the different versions of SAS}

\subsection{SAS 8.2}

\subsection{SAS 9.0}

\subsection{SAS 9.1 and beyond.}

\section{Summary}

\chapter{Using The HTML Tagsets}
There are several HTML tagsets to choose from. Each generates
a slightly different type of HTML.  Each have their advantages.
This chapter will explain the different HTML tagsets and the
features they support.
\section{The html statement}

\subsection{HTML4}

\subsection{PHTML}

\subsection{HTMLCSS}

\subsection{MSOffice2K}

\section{CHTML}

\section{Stylesheets}

\section{Javascript code}

\section{Scrolling Tables}

\section{Accessibility}

\section{HTML and Excel}

\section{Summary}

\chapter{Using Markup output with Spreadsheets.}
There are various ways to get SAS output into a spreadsheet.
Each have their advantages and disadvantages.  This chapter
will go over the various tagsets that are good for importing into
spreadsheet programs.
\section{Using CSV}

\section{SYLK}

\section{HTML and Excel}

\subsection{Compact HTML}

\subsection{PHTML - Plain HTML}

\subsection{HTML for Microsoft Office 2000}

\section{DDE}

\section{Spreadsheet XML}

\section{Summary}

\chapter{Using Tagsets with the Libname XML Engine}
The libname XML Engine also uses tagsets.  all of the output generated
from the XML engine is done through a tagset.  The libname engine's 
tagsets differ somewhat from the tagsets used by ODS Markup.  This chapter
will explain how to use tagsets with the libname engine and how they
differ from those used by ODS.
\section{XML Engine vs. ODS}

\section{Advantages of the XML engine}

\subsection{Control options}

\section{The Tagsets}

\section{Summary}

\backmatter
\part{Appendices}
\appendix
\begin{appendix}
\chapter{Quick Reference Guide}
\section{Useful tagsets}
\section{Tagset attributes}
\section{Event attributes}
\section{Event Statements}
\section{If Statements}
\chapter{Variables}
\section{Event Variables}
\subsection{508 Accessibility}
\subsection{Data}
\subsection{Event MetaData}
\subsection{Data Formatting}
\subsection{Title and Note Formatting}
\subsection{Miscellaneous}
\subsection{Graph}
\subsection{ODS Statement}
\subsection{Table}
\subsection{URL}
\subsection{XML Libname Engine}
\section{Style Variables}
\subsection{Borders}
\subsection{Font}
\subsection{Images}
\subsection{Layout}
\subsection{Miscellaneous}
\subsection{Text}
\chapter{Extended Examples}
\end{appendix}
\end{document}
