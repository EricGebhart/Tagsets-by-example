\chapter{File Redirection}
\index{File Redirection}
Adding output to the all the files specified on the ODS 
statement requires that some events be redirected to those
files.  This chapter will show the differences between the
various files and the events associated with them.  It will
also show how to use file redirection to modify their contents.

Most event requests are sent to the body file specified on
the ods statement.  The other files only get a skeletal number
of events.  The stylesheet file is an exception to that.  A
complete stylesheet file can be created with an ods statement
that looks like this.

\begin{sfvcode} 
ods html stylesheet="style.css";
\end{sfvcode} 

\index{Tagset!OdsStyle}
This statement gives an error, but the stlyesheet is still created.
There is one tagset that was meant to be used this way.  The odsstyle
tagset only defines stylesheet events.  Even if a body file is specified,
nothing will be written to it.

Creating a table of contents is one of the more common desires, yet the
contents file receives none of the event requests needed to create one.
This is great for document types like latex and XML that would rather 
have the hierarchical information embedded in the body of the file.
It is also not difficult to create a separate file if desired.


\section{The File Attribute}
Each event can have a file attribute that redirects which file the event's
output will go to.  The attribute values correspond to the names given on the
ods statement.
\begin{tabular}{l|l}
Value & File / Action \\ \hline
Contents & The contents file \\
Pages  & The pages file   \\
Frame  & The frame file   \\
Code  & The Code file   \\
Data  & The Data file   \\
Stylesheet  & The Stylesheet file \\
Body  & The body file \\ \hline
\end{tabular}

Once an event has been redirected all events triggered will also be redirected
until the redirected event is done.

\section{Identify}
\label{outline_map}
Running the short\_map tagset will show that there are several events that
give structure to the output.  The structure is even easier to see if the table
events are eliminated.  The new tagset looks like this.
It might have been easier to define the desired events instead of using the
default.  But this way, it's a sure bet nothing important has been eliminated.
For variety, this job uses the GLM procedure.  This procedure creates a fairly
complex table of contents.  The ODS select statement has been used to shorten
the output considerably.

\begin{sfvcode}
proc template;
    define tagset tagsets.leaf;
       parent = tagsets.short_map;
       image_formats = "gif,jpeg";
           
       define event data; end;
       define event row; end;
       define event table_body; end;
       define event table_head; end;
       define event cell_is_empty; end;
       define event rowspec; end;
       define event colgroup; end;
       define event colspecs; end;
       define event colspec_entry; end;
       define event colspecsep; end;
       define event header_spec; end;
       define event span_header_colspec; end;
       define event span_group; end;
       define event row_group; end;
       define event sub_rowheader_colspec; end;
       define event row_header_spec; end;
       define event col_header_label; end;
       define event cellspec; end;
       define event cellspecsep; end;
       define event doc_head; end;
       define event doc_meta; end;   
       define event auth_oper; end;   
       define event doc_title; end;
       define event stylesheet_link; end;
       define event code_link; end;
       define event javascript; end;
       define event startup_function; end;
       define event shutdown_function; end;
       define event table_headers; end;
       define event col_header_spec; end;
       define event sub_header_colspec; end;
       define event sub_colspec_header; end;
       define event header; end;
       define event cellspecspan; end;
       define event colspanfillsep; end;
       define event colspanfill; end;
       define event rowspancolspanfill; end;
       define event rowspanfill; end;
       define event system_title_setup_group; end;
       define event system_title_setup; end;
       define event system_footer_setup_group; end;
       define event system_footer_setup; end;
       define event title_setup_format_section; end;
       define event title_format_section; end;
       define event page_setup; end;
       define event title_setup_container; end;
       define event title_setup_container_specs; end;
       define event title_setup_container_spec; end;
       define event title_setup_container_row; end;
       define event title_container_row; end;
       define event title_container_specs; end;
       define event title_container_spec; end;
       define event put_value; end;
   end;
run;

   data plants;
      input type $ @;
      do block=1 to 3;
         input stemleng @;
         output;
         end;
      cards;
   clarion  32.7 32.3 31.5
   clinton  32.1 29.7 29.1
   knox     35.7 35.9 33.1
   oneill   36.0 34.2 31.2
   compost  31.8 28.0 29.2
   wabash   38.2 37.8 31.9
   webster  32.5 31.1 29.7
   ;

   data mileage;
      input mph mpg @@;
      cards;
   20 15.4 30 20.2 40 25.7 50 26.2 50 26.6 50 27.4 55  . 60 24.8
   ;

options ls=100;

ods select  GLM.Data.ClassLevels;
ods select  GLM.Means.type.stemleng.MCLines.Waller.MCLinesInfo;
ods select  GLM.Means.type.stemleng.MCLines.Waller.MCLines;


title 'A system title';
title2 'Another system title';
footnote 'A system Footer';
footnote2 'Another system Footer';

ods tagsets.event_map file="outline2.xml";
ods tagsets.leaf file="outline.xml";

   proc glm order=data data = plants;
      class type block;
      model stemleng=type block / solution;
      means type / waller regwq;

   *-type-order------------clrn-cltn-knox-onel-cpst-wbsh-wstr;
      contrast 'compost vs others'  type -1 -1 -1 -1  6 -1 -1;
      contrast 'river soils vs.non' type -1 -1 -1 -1  0  5 -1,
                                    type -1  4 -1 -1  0  0 -1;
      contrast 'glacial vs drift'   type -1  0  1  1  0  0 -1;
      contrast 'clarion vs webster' type -1  0  0  0  0  0  1;
      contrast 'knox vs oneill'     type  0  0  1 -1  0  0  0;
   run;
    
ods _all_ close;    
\end{sfvcode}

The output from this tagset is very interesting.  The most obvious
events are the proc, proc\_branch, branch and leaf events.  It is 
those last three that are used by the html tagsets to create the
table of contents.

\begin{sfvoutput}
<?xml version="1.0" encoding="iso-8859-1"?>

<doc operator="eric" sasversion="9.1" 
     saslongversion="9.01.01M0D10062003" 
     date="2003-11-17" time="01:59:01" encoding="iso-8859-1">
  <doc_body>
    <proc name="GLM">
      <anchor name="IDX"/>
      <system_title_group>
        <title_container>
          <system_title value="A system title">
          </system_title>
          <system_title value="Another system title">
          </system_title>
        </title_container>
      </system_title_group>
      <proc_title_group>
        <proc_title value="The GLM Procedure"/>
      </proc_title_group>
      <proc_branch name="GLM" value="GLM">
        <branch name="Data" value="Data">
          <leaf name="ClassLevels" value="Class Levels">
            <page_anchor/>
            <output name="ClassLevels">
              <table>
              </table>
            </output>
          </leaf>
        </branch>
        <system_footer_group>
          <title_container>
            <system_footer value="A system Footer">
            </system_footer>
            <system_footer value="Another system Footer">
            </system_footer>
          </title_container>
        </system_footer_group>
        <pagebreak/>
        <anchor name="IDX1"/>
        <page_anchor/>
        <system_title_group>
          <title_container>
            <system_title value="A system title">
            </system_title>
            <system_title value="Another system title">
            </system_title>
          </title_container>
        </system_title_group>
        <proc_title_group>
          <proc_title value="The GLM Procedure"/>
          <proc_title/>
          <proc_title value="Waller-Duncan K-ratio t Test for stemleng"/>
        </proc_title_group>
        <branch name="Means" value="Means">
          <branch name="type" value="type">
            <branch name="stemleng" value="stemleng">
              <branch name="MCLines" value="Multiple Comparison Lines">
                <branch name="Waller" value="Waller-Duncan">
                  <leaf name="MCLinesInfo" value="Information">
                    <output name="MCLinesInfo">
                      <table>
                      </table>
                    </output>
                  </leaf>
                  <anchor name="IDX2"/>
                  <leaf name="MCLines" value="Comparisons">
                    <output name="MCLines">
                      <table>
                      </table>
                    </output>
                  </leaf>
                </branch>
              </branch>
            </branch>
          </branch>
        </branch>
        <system_footer_group>
          <title_container>
            <system_footer value="A system Footer">
            </system_footer>
            <system_footer value="Another system Footer">
            </system_footer>
          </title_container>
        </system_footer_group>
      </proc_branch>
    </proc>
  </doc_body>
</doc>
\end{sfvoutput}

The events that can be seen in this map are the basis for all document
structure.  Redirecting these events or triggering other events that
are redirected allows great flexibility in the creation of a table of
contents or document outlines.  Using event\_map as the parent to this
tagset would reveal more variables and values which could also be useful.

Among the variables that could be seen are toclevel, and url.
Toclevel is a counter that indicates the contents depth.  Url is 
the url for the current output object.  

\section{Locate and Explore}
\index{Event!Branch}
\index{Table of Contents}
\index{Table of Contents!Flattening}
The next step is to look in the tagset source to see what these events
are used for.  The HTML tagsets do very little in the proc\_branch, branch 
and leaf events. These events check the 'hidden' variable and trigger other events 
that are redirected to the 
contents file.  An obvious and easy modification would be to block
the branch event.  The result would be a very flat table of contents.
Depending on the Procedure, this is sometimes desirable.

\begin{sfvcode}
proc template;
    define tagset tagsets.flat_contents;
        parent=tagsets.html4;
 
        define event branch;
        end;
    end;
run;
\end{sfvcode}

Another variation would be to block the branch entries based on the toclevel
variable.  It is possible to eliminate any levels desired.  This example eliminates
contents levels deeper than 3.

\begin{sfvcode}
proc template;
    define tagset tagsets.flat_contents;
        parent=tagsets.html4;
 
        define event branch;
            start:
                break / if hidden;
                break / if inputn(toclevel, '7.') > 3;
                trigger contents_branch;
            finish:
                break / if hidden;
                break / if inputn(toclevel, '7.') > 3;
                trigger contents_branch;
        end;
    end;
run;
\end{sfvcode}

Another interesting event is the page\_anchor event.  This event is redirected
to the pages file to create links to each page of output.  It might be desirable
to redirect the page\_anchor event to the contents file.  This tagset will do that.
This tagset preserves the original behavior by creating a new event from the original
page\_anchor event.  Page\_anchor\_entry is exactly the same but without the file 
redirection.  There are two new events whose sole purpose is to redirect the output
from the new event to the appropriate file.  The end result of this tagset is that
page numbers are inserted as links in the table of contents.

\begin{sfvcode}
proc template;
    define tagset tagsets.flat_contents;
        parent=tagsets.html4;
 
        define event page_anchor;
            start:
                trigger contents_page_anchor;
                trigger pages_page_anchor;
            finish:
                trigger contents_page_anchor;
                trigger pages_page_anchor;
        end;

        define event contents_page_anchor;
            file=contents;
            start:
                trigger page_anchor_entry;
            finish:
                trigger page_anchor_entry;
        end;

        define event pages_page_anchor;
            file=pages;
            start:
                trigger page_anchor_entry;
            finish:
                trigger page_anchor_entry;
        end;

        define event page_anchor_entry;
            start:
                put "<li";
                putq " class=" HTMLCLASS;
                put ">" nl;
                put '<a href="' ;
                put path_name /if !exists(path_url);
                put path_url;
                put body_name /if !exists(body_url);
                put body_url;
                put "#" ANCHOR;
                put '"';
                putq ' target="body" ';
                put ">";
                trigger pre_post;
                put 'Page ' total_page_count;
                trigger pre_post finish;
                put "</a>" CR;
            finish:
                put "</li>" nl;
        end;
    end;
run;
\end{sfvcode}

\section{File interactions}
The ODS markup statement allows for seven separate file specifications.  Each of these 
files has certain interactions with the others.  Specifying a stylesheet or code file
will result in stylesheet\_link and code\_link event requests being sent to the body,
contents, and pages files.  A frame file will receive event requests providing information
about the body, contents, and pages files if they were specified.
The easiest way to see all of the file interactions is to run a job like this one.
The event\_map tagset may also be helpful.

\begin{fvcode}{allmaps}{Getting simple maps for all files}
ods tagsets.short_map file='body.xml' 
                  contents='contents.xml' 
                      page='pages.xml' 
                     frame='frame.xml' 
                stylesheet='style.xml' 
                      code='code.xml' 
                      data='data.xml';
ods tagsets.short_map close;
\end{fvcode}

Looking at the maps created for each file will reveal the various places each file is
referenced by another.

\begin{enumerate}
\item The stylesheet is referenced to by every file except the frame file.
\item The code file is referenced to by the body, contents, and pages files.
\item The frame file includes frame events for the body, contents and pages file.
\end{enumerate}

\section{Freedom of Choice}
\index{Variables!Filename}
While there are events that are typically used to reference the other files,
it is not necessary to use them to do so.  All of the filenames and their url's
are available at all times.  The following event finish definition is from the html 
tagsets.  It uses the file names and urls to create a noframe section for
508 compliance.

\begin{sfvcode}
        finish:
          put "</frameset>" nl;
          put "<noframes>" nl;
          put "<ul>" nl ;

          do /if any(contents_name,contents_url);
              put '<li><a href="';
              put basename "'";
              put path_url;
              put path_name /if ^path_url;
              put "(";
              put contents_url;
              put contents_name /if ^contents_url;
              put ")'" '"';
              put ">The Table of Contents</a></li>" nl;
          done;

          unset $link;
          do /if any(pages_name,pages_url);
              put '<li><a href="';
              put basename "'";
              put path_url;
              put path_name /if ^path_url;
              put "(";
              put pages_url;
              put pages_name /if ^pages_url;
              put ")'" '"';
              put ">The Table of Pages</a></li>" nl;
          done;

          put '<li><a href="';
          put basename "'";
          put path_url;
          put path_name /if ^path_url;
          put "(";
          put body_url;
          put body_name /if ^body_url;
          put ")'" '"';
          put ">The Contents</a></li>" nl;

          put "</ul>" nl ;
          put "</noframes>" nl;

          unset $link;
      end;
\end{sfvcode}

\subsection{Explore}
Like most other things, A tagset and ods statement can be written that explores
these variables.  The following job will show the file variables that are available. 

\begin{fvcode}{exploring_filenames}{Exploring Filename Variables}
proc template;
    define tagset tagsets.filenames;
        define event initialize;
            putvars event _name_ ': ' _value_ nl;
        end;
    end;
run;

ods tagsets.filenames file='body.xml'  (url='b.html')
                  contents='contents.xml' (url='c.html') 
                      page='pages.xml' (url='p.html')
                     frame='frame.xml' (url='f.html')
                stylesheet='style.xml' (url='s.html')
                      code='code.xml'  (url='cd.html')
                      data='data.xml'  (url='d.html');
ods tagsets.filenames close; 
\end{fvcode}

\index{Variables!Filename}
Because the initialize event only happens once, it happens in the 
first file that is opened.  That happens to be the code file.  All 
the other files created are empty.  But the output is quite interesting.
Most of the variables are populated at all times.

\begin{sfvoutput}
event_name: initialize
empty: 1
encoding: iso-8859-1
operator: eric
date: 2003-11-17
sasversion: 9.1
saslongversion: 9.01.01M0D10062003
time: 21:28:53
state: start
dest_file: code
bodyname: body.xml
bodyurl: b.html
contentsname: contents.xml
contentsurl: c.html
pagesname: pages.xml
pagesurl: p.html
stylesheet: style.xml
stylesheet_url: s.html
codename: code.xml
codeurl: cd.html
dataname: data.xml
dataurl: d.html
framename: frame.xml
frameurl: f.html
stylesheet_title: style.xml
tagset: TAGSETS.FILENAMES
style: Default
javadate: 2003-11-17
javatime: 21:28:53-05:00
\end{sfvoutput}

\section{Summary}
File redirection is an important feature of tagsets.  Without it there would be no
separate table of contents.  But redirection can be used for more than just contents.
The data and code files were created specifically to allow for separation of data, code
and formatting if that is desired.  

