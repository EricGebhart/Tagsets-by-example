\chapter{Modifying Existing Tagsets}
\index{Tagset!modifying}
\index{Tagset!inheritance}
The most common tagset programming is done to
modify the behavior of an existing tagset. 
Modifying a tagset is fairly easy to do because
tagsets can inherit events from each other.  
It is also easy to identify which events need changing.
This chapter will introduce a few new concepts
while showing the steps needed to modify a tagset
through inheritance.  
 

\index{CSV}
\section{Changing the delimiter for CSV files}

It is a common request to change the delimiter used
by the CSV destination.
A common alternative to the comma is the semi-colon, another
popular choice is tab. These examples are going to
show how to create a new tagset, that does this
through inheritance.  The CSV tagset is not as simple
as it once was.  The difference between the current CSV
tagset and the original CSV tagset that shipped with SAS 8.2
is like night and day.  Despite it's surprising complexity
the CSV tagset is still easy to derive other tagsets from.

With the version from SAS 8.2, creating a tagset that used
a different delimiter was actually more complex.  The current
version actually has an option that we can specify on the 
ODS statement or within a new tagset that we define ourselves.

Because it is a good example, I will take you through the process
of modifying the CSV tagset from SAS 8.2.  If you need it this
tagset is readily available at http://support.sas.com/rnd/base/topics/odsmarkup.

After that, we'll go through the motions with the CSV tagset available
for SAS 9.1.3 dated April 2006.


\index{Finding Events}
\subsection{Finding the Events}
The first thing to do is find the events that
put the commas in between the comma separated values.
A good guess is that the data and header events will
be involved.  But one of the nice things about inheriting
from another tagset is that it can be examined for clues.
Since this tagset is going to modify the csv tagset, that
is the tagset to inherit from.
There are two ways to look at the csv
tagset.  Proc template is my least favorite, because
it loses white space and comments.  It also reverses
the order of the tagset attributes and events.  But
to use it do the following.

\begin{fvcode}{Source_tagsets}{Getting the CSV Source}

 proc template;
   source tagsets.csv;

\end{fvcode}

My favorite way to look at tagsets is by using the original
source which is downloadable from the SAS web site.  The source
can also be examined in the template browser or by using the
source statement in proc template.  However, those methods do
not show white space or comments that are in the original source.
That makes the original tagset source code far superior to the
tagset source you can see through the template interface.

Looking at the source 
in Figure \ref{comma_events} on page \pageref{comma_events} 
shows all the events which have commas.

\begin{fvcode}{comma_events}{CSV Events with commas}

        define event header;
          start:
            put ',' / if !cmp(COLSTART, "1");
            put '"';
            put VALUE;
          finish:
            put '"';
        end;

        define event data;
          start:
            put ',' / if !cmp(COLSTART, "1");
            put '"';
            put VALUE;
          finish:
            put '"';
        end;

        define event colspanfill;
            put ',';
        end;

        define event rowspanfill;
            put ',' /if ! exists(VALUE);
        end;


\end{fvcode}

\subsection{Simple Ifs}
\index{if}
\index{compare}
\index{cmp}
\index{exists}
\index{any}
\index{contains}
\index{if!compare}
\index{if!cmp}
\index{if!exists}
\index{if!any}
\index{if!contains}
These events are readily understandable, but contain
three if statements that need explanation.  This type
of if only applies to the statement it is connected
to.  There are 5 tests that are extremely 
efficient.  The first two are, cmp() and exists().
The others are any(), contains(), and a variable by its
self. Exists, Any, and the simple
variable tests evaluate for a non-zero length of strings
and the actual value of numeric variables.  A value of 0 
is false, non-zero is true.

\subsubsection{Compare}
\index{compare}
\index{cmp}
\index{variables!colstart}
Both of these tests are preceded
by an exclamation point.\footnote{An ! can only occur
at the beginning of a test}  An exclamation point or
caret means not.  Cmp is an effient case insensitive
compare.  The variable colstart always contains the 
number of the column that the current cell starts in.

The data and header events, use cmp to suppress the comma,
on first value of the row.

\subsubsection{Exists}
\index{exists}
The final if in the tagset is an exists.  Exists can take any number
of variables as arguments.  If all the variables have
a value then exists is true, otherwise it is false.
The following pseudo code shows how exists can be thought 
of as a group of ands. 
\begin{sfvcode}
  if exists(variable1, variable2, variable3)
  if variable1 and variable2 and variable3
\end{sfvcode}
In a case like this, the variable could be listed
by its self like this.
\begin{sfvcode}
  put ',' /if ! exists(variable);
  put ',' /if ! variable;
\end{sfvcode}
Either syntax will cause 
the put in the rowspanfill event will print a comma
if the VALUE variable does not have a value.  Rowspanfill is one of those
elusive events that is is rarely sighted.  The best way to coax it out of
hiding is to use Proc Tabulate.

\subsubsection{Any}
\index{any}
Any() is just like exists() but is true if any one
of the variables given has a value.  Think of it as
a group of OR's as demonstrated by this pseudo code.
\begin{sfvcode}
   if any(variable1, variable2, variable3)
   if variable1 or variable2 or variable3
\end{sfvcode}

\section{Making the changes}
Even with the new syntax just discussed, the job is pretty easy.  All that
is desired is to change those commas to semi-colons.

Tagset inheritiance is very straight forward.  A tagsets inherits all attributes
and events from the parent tagset.  Any changes require a complete redefinition of
the event to be changed.

The new tagset only needs to specify tagsets.csv as it's parent and redefine those
four events with semi-colons instead of commas.  The new tagset is shown in 
listing \vref{Semi-csv}.


\begin {fvcode}{Semi-csv}{Semi-Colon separated values}
    Define tagset tagsets.SSV
        parent=tagsets.csv;
    
        define event header;
            start:
                put ';' / if !cmp(COLSTART, "1");
                put '"';
                put VALUE;
            finish:
                put '"';
        end;

        define event data;
            start:
                put ';' / if !cmp(COLSTART, "1");
                put '"';
                put VALUE;
            finish:
                put '"';
        end;

        define event colspanfill;
            put ';';
        end;    

        define event rowspanfill;
            put ';' /if ! exists(VALUE);
        end;

     end; 
\end{fvcode}

Running the following proc print 
with the new tagset gives us the output.
Shown in figure \vref{semi_output}.
\begin{sfvcode}
options obs=3;
ods tagsets.ssv file='test.ssv';
proc print data=sashelp.class;run;
ods tagsets.ssv close;
\end{sfvcode}

\begin{poutput}{semi_output}{Semi Colon Separated output}
"Obs";"Name";"Sex";"Age";"Height";"Weight"
" 1";"Alfred";"M";"14";"69.0";"112.5"
" 2";"Alice";"F";"13";"56.5";" 84.0"
" 3";"Barbara";"F";"13";"65.3";" 98.0"
\end{poutput}
       
\section{A better CSV tagset}
\index{Define!event}
\index{Event!data}
This new but old tagset is nice, but it's not very flexible.  It's not that
hard to write a new one if a different delimiter is desired, but wouldn't
it be nice to have a tagset that would use any delimiter it was given?
Macro variables are one way to achieve more flexibility.  Once declared
they can be used anywhere within the tagset.  Macro variables in a tagset
can get out of hand.  Since SAS 9.1.3 there has been an options option that
gives the same flexibility without cluttering up the SAS macro name space.

\subsection{Macro variables}
\index{Macro Variables}
Macro variables are one way to pass values into a tagset.
Any macro variable can be accessed
within the tagset as long as it is declared.  A macro
variable can be used to define the delimiter.  A comma 
can be used in its absence.
A macro variable is declared in a tagset like this:
\begin{sfvcode}
  mvar myvariable;
  nvar myOtherVariable;
\end{sfvcode}
As long as they are declared the tagset will know what to do when
they are used.  If they are not declared then the tagset will fail to
compile when you reference them later.

\subsection{The Initialize Event}
\index{Event!Initialize}
The code could use the macro variable everywhere there is a comma, but
that would require an if at each point to determine if a comma 
was needed instead.  That would add unecessary complexity.  It
would also allow the delimiter to be changed in midstream.  A changing
macro variable might be good in other situations, but it's not very 
useful here.

It would be best to set up the delimiter once at the very beginning
and use that everywhere.  There is a special event especially for
this sort of thing.  The Initialize event\footnote{Only available in SAS 9.0
or greater, use the doc event in earlier releases} happens once when the tagset
is first invoked.  All that is needed is to examine the macro variable
and create a tagset variable accordingly.

\subsection{The set statement}
\index{Set}
\index{Statements!Set}
By using the Set statement\footnote{Available in SAS 9.0 and greater}
it is possible to create variables within a tagset.  All variables
are global and will persist until they are unset or the output destination
is closed.  Set will be used to create a new delimiter variable that
will be populated with the macro variable or, lacking that, a comma.

\subsection{The New CSV Tagset}        
Our new old school (SAS 8.2) tagset is shown in figure \vref{csv_1}.

\begin{fvcode}{csv_1}{A Better CSV Tagset}
proc template;

    Define tagset tagsets.SSV;
        parent=tagsets.csv;

        mvar csv_delimiter;

        define event initialize;
           set $delimiter csv_delimiter;
           set $delimiter ',' /if !$delimiter;
        end;

        define event header;
            start:
                put $delimiter / if !cmp(COLSTART, "1");
                put '"';
                put VALUE;
            finish:
                put '"';
        end;

        define event data;
            start:
                put $delimiter / if !cmp(COLSTART, "1");
                put '"';
                put VALUE;
            finish:
                put '"';
        end;

        define event colspanfill;
            put $delimiter;
        end;    

        define event rowspanfill;
            put $delimiter /if ! exists(VALUE);
        end;

     end; 
run;

%let csv_delimiter=|;

options obs=3;
ods tagsets.ssv file='test.ssv';
proc print data=sashelp.class;run;
ods tagsets.ssv close;
\end{fvcode}

For this run the delimiter has been set to '|' which causes
the output to appear as in figure \ref{csv_1_out} on page \pageref{csv_1_out} 

\begin{poutput}{csv_1_out}{Semi Colon Separated output}
"Obs"|"Name"|"Sex"|"Age"|"Height"|"Weight"
" 1"|"Alfred"|"M"|"14"|"69.0"|"112.5"
" 2"|"Alice"|"F"|"13"|"56.5"|" 84.0"
" 3"|"Barbara"|"F"|"13"|"65.3"|" 98.0"
\end{poutput}
       
This tagset is quite flexible.  But there's the next best thing that will
make it even more flexible.  Macro variables are nice, but
they can be hard to keep track of.  

\subsection{Tagset Alias}
\index{Alias}
\index{tagset\_alias}
\index{Variables!tagset\_alias}
At some point there was a realization that a tagset might want to behave
differently on occasion.  That it would assume an alias, a sort of alternate
personality.
from it's usual behavior.   What I didn't realize at that time is that the
future would call for a multiply split personality and that a simple alias 
would fall quite short of providing the nuances we would later need.

The ods markup statement has an alias option that was created just for
this sort of purpose.
It works very simply, any value assigned to alias becomes
the value of the tagset\_alias variable in the tagset.  All that is needed
is to add it to the initialize event.  Once that happens, the delimiter can
be specified on the ods statement.

For simplicity this example creates a completely new tagset which inherits
from the example in \ref{csv_1} on page \pageref{csv_1}.  The only event that
needs to be changed is the initialize event.  There is no reason to repeat
all the other events.  Inheritance keeps the example short and simple.
In reality this would probably be incorporated into one tagset.

The delimiter should be the value specified on the ods statement first.  The
value of the macro variable second, and if neither of those are specified
then the delimiter should default to comma.
The resulting code is shown in figure \ref{csv_2} on page \pageref{csv_2}.
In this example the macro variable is still set to '|' but an ! is given
on the ods statment.  
The resulting output is shown in figure \ref{csv_2_out} on page 
\pageref{csv_2_out}.

\begin{fvcode}{csv_2}{Using alias}
proc template;

    Define tagset tagsets.SSV2;
        parent=tagsets.ssv;

        mvar csv_delimiter;

        define event initialize;
           set $delimiter tagset_alias;
           set $delimiter csv_delimiter /if !$delimiter;
           set $delimiter ',' /if !$delimiter;
        end;
     end; 
run;

%let csv_delimiter=|;

options obs=3;
ods tagsets.ssv2 file='test.ssv' alias='!';
proc print data=sashelp.class;run;
ods tagsets.ssv2 close;

\end{fvcode}

\begin{poutput}{csv_2_out}{Variable delimited output}
"Obs"!"Name"!"Sex"!"Age"!"Height"!"Weight"
" 1"!"Alfred"!"M"!"14"!"69.0"!"112.5"
" 2"!"Alice"!"F"!"13"!"56.5"!" 84.0"
" 3"!"Barbara"!"F"!"13"!"65.3"!" 98.0"
\end{poutput}

\subsection{Tagset Options}
\index{Options}
\index{\$options}
\index{Variables!\$options}
As of SAS 9.1.3 the ODS markup destination has a new option.  Options can
now be passed to the tagset with the options(...) option.  This allows 
great flexibility in the tagset with a central control in the ods 
statement.  Options are much more powerful than the Alias option, and
much easier than macro variables.  As a tagset becomes more complex it
is often necessary to give it more controls.  Macro variables can get
out of hand and become hard to manage.  As of this writing, the most 
current CSV tagset has several options.  Other tagsets have even more.

This new tagset, in Figure \ref{csv_2} on page \pageref{csv_2}, is 
a great place to start when creating any new tagset.  As new options
are added the help can be updated at the same time.  All the parts
are here to create options and documentation as you go.

\begin{sfvcode}
ods tagsets.ssv2 file='test.ssv' options(delimeter='!' doc='help');
\end{sfvcode}

Any name/value pairs given within the parenthesis become entries in a
tagset dictionary variable called \$options.  It is up to the tagset
to use them.  It is also the responsibility of the tagset
to document their use.  


\begin{fvcode}{csv_2}{Using alias}
proc template;

    Define tagset tagsets.SSV2;
        parent=tagsets.ssv;

        mvar csv_delimiter;

        define event initialize;
           set $delimiter tagset_alias;
           set $delimiter $options['DELIMITER'] /if $options;
           set $delimiter csv_delimiter /if !$delimiter;
           set $delimiter ',' /if !$delimiter;
           
           trigger help /if $options;
        end;

        define event options_set;
           trigger initialize;
        end;

        define event help;
           break /if ^cmp($options['DOC'], 'help');
           putlog ' ';
           putlog 'Help for the SSV2 tagset';
           putlog ' ';
           putlog 'The following options are available';
           putlog 'doc: No default value.';
           putlog '     help - this text.';
           putlog ' ';
           putlog "delimiter: default value: ','";
           putlog '     The value of this option will be used';
           putlog '     as the delimeter for the data printed.';
           putlog ' ';
        end;
      
     end; 
run;

%let csv_delimiter=|;

options obs=3;
ods tagsets.ssv2 file='test.ssv' options(delimiter='!' doc='help');
proc print data=sashelp.class;run;
ods tagsets.ssv2 options(delimeter=';');
proc print data=sashelp.class;run;
ods tagsets.ssv2 close;

\end{fvcode}

This tagset introduces a new event called Options\_set.  The Options\_set
event will happen when a subsequent ods statement changes the options
on an ods statement that is already open.  The second ods statement in
the example exercises the Options\_set event.


\section{The Current Methodology}

\index{CSV}
All of these examples have been great for getting an understanding of the
CSV tagset, events, if statements, macro variables, Tagset Alias, and the
Options option.  But how would you do all of this now?  With the CSV tagset
in all of it's modern complexity.  Well, that's not so hard as all of this.

\index{Options}
\index{Options!CSV}
\index{\$options}
In it's current version, the CSV tagset has an option for the delimiter. 
And another for turning on titles,  and another for footnotes, bylines, 
procedure titles, and notes.  Even for turning column headers on and off
or changing the way currency values are treated.  That can lead to a rather
long set of options on the ods statement.

What if you wanted CSV output that had titles but no table headers, and it
treated money like numbers.  And we wanted a semi-colon as a field separator.
Here is the ods statement we could use.

\index{Options!CSV}
\begin{sfvcode}
    ods csv file='test.csv'
            options(delimiter=';' titles='yes' Currency_as_number='yes');
\end{sfvcode}

\index{Options!CSVALL}
\index{Options}
The interesting thing about this is that there are other CSV tagsets.  CSVALL
and CSV\_byline don't really do much now.  With SAS 8.2 they actually had a
fair amount of code in them.  Currently all they do is set up some options.
They are a great example of how we can create a new destination that behaves
quite differently from the tagset it was derived from.  But without adding 
much code at all.

Here is the entire source code for the CSVALL tagset.

\index{Options!CSVALL}
\index{Options!Set}
\index{Set}
\index{Statements!Set}
\begin{sfvcode}
    define tagset tagsets.csvall;
        parent= tagsets.csv;
        notes "This is the CSV with titles and bylines definition";

        define event initialize;
            set $options['BYLINES'] 'yes';
            set $options['TITLES'] 'yes';
            set $options['PROC_TITLES'] 'yes';
            set $options['NOTES'] 'yes';
            trigger set_options;
            trigger documentation;
            trigger compile_regexp;
        end;

    end;
\end{sfvcode}

\index{Options!CSVALL}
\index{Options!Set}
\index{Set}
\index{Statements!Set}
\index{CSV!Delimiter}
This can be used as a template for other CSV tagsets.  To create this
tagset the initialize event was copied from the parent CSV tagset.
Set statements were added to create a tagset that would over ride any
settings or lack of settings on the ods statement.  In this case the
options for bylines, titles, proc titles and notes are all turned on.
It would be just as simple to create a new tagset that changed the default
delimiter.  

In this example we want to test for an incoming value before
setting it to something else.  That is actually something that might
be nice in the CSVALL tagset.  As it stands, the CSVALL tagset would
not allow for the titles to be turned off.  It would set that option
to yes regardless of the options on the ods statement.  Here is the
new ssv tagset.

\index{Options!CSVALL}
\index{Options!Set}
\index{Set}
\index{Statements!Set}
\index{CSV!Delimiter}
\begin{sfvcode}
    define tagset tagsets.ssv;
        parent = tagsets.csv;
        notes "This is the CSV with titles and bylines definition";

        define event initialize;
            do /if ^$options['DELIMITER'];
                set $options['DELIMITER'] ';';
            done;
            trigger set_options;
            trigger documentation;
            trigger compile_regexp;
        end;

    end;
\end{sfvcode}

I think you'll agree that tweaking tagset behavior has become easier 
although the tagsets themselves are more complex.  
It is important to be mindful of the future when writing 
tagsets.  Compartmentalization makes future enhancements much easier.

\section{Summary}
All of the things covered here will be covered in more detail later.
These few examples have covered a lot of ground.

Most of the idea's introduced here are used extensively 
in all tagsets.  Inheritance is the primary way of modifying and creating
new tagsets.  Simple if statements are the most common form of controlling
the program flow within events.  Macro variables, tagset alias, and the
options option are ways to affect the personality of a tagset from one 
usage to the next.

