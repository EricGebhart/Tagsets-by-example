\chapter{Procedural controls}
\index{control!procedural}
\index{control!procedural!do}
\index{control!procedural!else}
\index{control!procedural!done}
\index{control!procedural!break}
\index{control!procedural!breakif}
The tagset's if syntax has already been used in previous examples. 
With SAS 8.2 and 9.0 if statements were kept to a few simple 
choices.  This simple if statement is still the most efficient
and most frequently used type of conditional programming that
tagsets have.  With SAS 9.1 if statments have been extended with
where clauses and new statements to provide fully featured procedural
controls.  
The examples in this chapter will explain the if syntax and the procedural control
statements that complement them.

\section{Simple If's}
\index{if}
\index{when}
\index{where}
\index{/}
An if can be
placed at the end of almost any tagset statement\footnote{eval and done are 
the only statements that cannot have an if}
.  But it must
be preceded by a /.  Usually an {\itshape if} follows the /.  But that is optional.
In fact, the keywords {\itshape when} and {\itshape where} are also allowed.  But it's
all personal preference since none of the words mean anything.  There
are two cases where the keyword following the / is important. 
{\itshape Breakif} and {\itshape While} cause the if to do more than a simple test.

\subsection{Built in tests}
\index{if}
\index{compare}
\index{cmp}
\index{exists}
\index{any}
\index{contains}
\index{if!compare}
\index{if!cmp}
\index{if!exists}
\index{if!any}
\index{if!contains}
These four tests
are very tightly integrated and are therefore faster than any other
type of test.  Any of these can be preceded with
a ! or \^ which indicates negation of the result.  The ! can only be
used for 'not' at the very beginning of an if, while \^ can be used
at any time.  Table \ref{builtin_ifs} on page \pageref{builtin_ifs}
shows the forms of the builtin if tests.  One very important distinction
between these if tests is that the variables in the tests do not have
to be defined.  If a variable doesn't exist the tests will evaluate to
false.  This is very convenient for boolean variables.  Set the variable
to some string for true, and unset it for false.  On the other hand,
referencing a non-existant variable in a where clause frequently
causes runtime errors.

If a variable in a where clause is not defined then an error
will result.

\begin{table}\caption{Built in 'if' tests}
\label{builtin_ifs}
\begin{tabular}{l|l}\\ \hline
Cmp &  Case insensitive compare of two string values.\\
Exists & Check if all values have a length or are non-zero\\
Any & Check if any values have a length or are non-zero\\
Contains & Case sensitive search for string inside a string.\\
Variable & Is the string length or numeric value.
\end{tabular}
\end{table}

The following example uses each of these tests in use.

\begin{sfvcode}
        define event initialize;
           set $is_eric 'True' /if cmp(operator, 'eric');
           put 'Hello Eric' nl /if $is_eric;
           put 'Hello ' operator nl /if ^$is_eric;
           put 'RIC' nl /if contains(operator, 'ric');
           put 'This might be Eric' nl /if any(operator, $is_eric);
           put 'This is Eric' nl /if exists(operator, $is_eric);
        end;
\end{sfvcode}

\section{Where Clauses}
\index{if!Where clauses}
\index{Where clauses}
In addition to the simple if conditions
above if tests can also use where clauses.  It turns out that they
are usually used when doing numeric comparisons, but some problems reguire
more complex if's.  The where processing provides that.

\begin{sfvcode}
        define event initialize;
           eval $count 1;
           put 'Count is > 0' nl /if $count > 0;

           eval $max 10;
           put 'max > Count > 0 ' nl /if $count > 0 and $count < $max;
        end;
\end{sfvcode}


\section{Break}
\index{Statements!Break}
\index{Break}
Break is a statement just like put or set.  When encountered the program
flow exits the event immediately.  This is useful for skipping the end
of an event when the remaining statements are not desired.  Here is an
example based on the initialize event from the example in in 
figure \ref{csv_2_out} on page \pageref{csv_2_out}.

\begin{sfvcode}
        define event initialize;
           set $delimiter tagset_alias;
           break /if $delimiter;

           set $delimiter csv_delimiter /if !$delimiter;
           set $delimiter ','           /if !$delimiter;
        end;
\end{sfvcode}

This is more efficient when tagset\_alias is used because only one if
is executed instead of two.  But it is less efficient if neither tagset
alias or the csv\_delimiter macro variable is set because all three if's
will have to execute.  Efficiency is not so important in this case, but
it can be very important if the event is used very frequently.  A poorly
coded data event can have a significant impact on performance.


\section{Breakif}
\index{if}
\index{if!Breakif}
Breakif is one of the two special keywords that can be used to define an
if test on a tagset statement.  When looking through the tagset code, 
there are many times when it is desirable to do something and then break
if a certain condition is met.  With breakif it is possible to write this
with one line instead of two.  That means one test evaluation instead of
two.
Breakif\footnote{Available in SAS 9.1 and later} can be quite useful in
these situations because it combines break with the execution of another
command based on an if.  Using the same example using breakif looks like this:
\begin{sfvcode}
        define event initialize;
           set $delimiter tagset_alias  /breakif tagset_alias;
           set $delimiter csv_delimiter /breakif csv_delimiter;
           set $delimiter ',';
        end;
\end{sfvcode}

This is a nice, efficient, and readable way to control the program flow.  As
soon as any one of the if's is true the delimiter will be set and the event 
will be exited.

\section{Do blocks}
\index{Statements!Do}
\index{Statements!Else}
\index{Statements!Done}
\index{Do}
\index{Else}
\index{Done}
Do blocks\footnote{Available in SAS 9.1 and later} provide a way to 
group several event statements together under one condition.  
do blocks start with a Do statement, end with a Done statement and may
have any number of else statements in between.  Using the same example
with do else looks like this.
\begin{sfvcode}
        define event initialize;

           do /if tagset_alias;
               set $delimiter tagset_alias;

           else /if csv_delimiter;
               set $delimiter csv_delimiter;

           else;
               set $delimiter ',';
           done;

        end;
\end{sfvcode}
\index{Statements!Set}

This is the most efficient construct so far.  but can be refined further by
eliminating one more if.  If tagset\_alias has no value, the set does nothing,
so putting an if around it isn't really an improvement.  It comes down to 
readability and personal taste.

\begin{sfvcode}
        define event initialize;

           set $delimiter tagset_alias;

           do /if !$delimiter;
               set $delimiter csv_delimiter;

           else;
               set $delimiter ',';
           done;

        end;
\end{sfvcode}

\section{Do While Loops}
\index{Looping}
\index{While}
\index{if}
\index{if!While}
\index{Stop}
\index{Continue}
\index{Statements!Stop}
\index{Statements!Continue}
Using the do statement it is also possible to create loops.  The behavior of the do
block is changed to do looping with  
the help of another special keyword.  Like {\itshape breakif}, 
{\itshape while}\footnote{The while keyword is only valid on the Do statement}, 
has special meaning.
it causes the do block to loop as long as the while test is true.  A Do /while
block can also have an else, or even multiple else /if's.  If the initial while
test is false, execution will fall to the else.  But if the while block executes
at all then the else will not be executed.

\begin{sfvcode}
proc template;
define tagset tagsets.dowhile;
    default_event = 'count';

    define event initialize;
        eval $count 1;

        do /while $count <3;
            put "Count < 3 " $count nl;
            eval $count $count + 1;
        else;
            put "Else from: Count < 3: " $count nl;
        done;

        do /while $count <3;
            put "Second Count < 3:" $count nl;
            eval $count $count + 1;
        else;
            put "Second Else: Count >= 3: " $count nl;
        done;
        
    end;
end;
run;

ods tagsets.dowhile file="dowhile.txt";
ods tagsets.dowhile close;
\end{sfvcode}

The output shows a count of 1 and 2.  The second loop fails to 
execute and displays the message from the else.  The output follows.

\begin{sfvcode}
Count < 3 1
Count < 3 2
Second Else: Count >= 3: 3
\end{sfvcode}

To round out the looping controls there are stop and continue statements.  Stop
will cause the loop to stop, program flow will pick up at the done.  Continue
will cause program flow to start back at the top of the loop again.  Break
behaves as always and causes the entire event to be aborted.  It is very easy
to make infinite loops, be careful.

\begin{sfvcode}
      eval $i 0;

      put "i is " $i nl;

      put "Going into a Loop to 10" nl;
      put "Continue at 5" nl;
      put "stop at 8" nl;

      do /while $i < 10;

          eval $i $i+1;

          continue /if $i eq 5;
          stop     /if $i eq 8;

          put "I is " $i nl;

      else;
          put "do this if i started out > 10" nl;

      done;
\end{sfvcode}

\section{Iterating through Dictionaries}
\index{Lists}
\index{Dictionaries}
\index{Looping}
\index{While}
\index{Iterate}
\index{Next}
\index{Statements!Iterate}
\index{Statements!Next}
Using do loops, the capability of iterating through a dictionary of values would be a 
very nice thing
to do.  Looping through a list is easy, all that is needed is a counter.  But dictionary's are
not so easy.  To loop through a dictionary there are iterate and next statements.  
Iterate and next work similarly to putvars in that they populate the special variables
\_name\_ and \_value\_.  Iterate
sets up the iterator for us and populates the first name and value into \_name\_ and \_value\_
respectively.  From then on, only the next statement is used.  Of course an iterator will
work on a list too, but there won't be a \_name\_.  These examples are awfully simple, putvars 
could have done this in one line.  The output can be seen in figure \ref{controls_out} on page
\pageref{controls_out}.

\begin{sfvcode}

      set $dogs['Ernie'] 'Was a great dog';
      set $dogs['Pola'] 'Was really sweet.';
      set $dogs['Arkas'] 'Is a big dog';
      set $dogs['Luna'] 'Is super sweet';

      iterate $dogs;
      do /while _value_;
          put "Key: " _name_ " Value: " _value_ nl;
          next $dogs;
      done;

      set $dog_keys[] 'Ernie';
      set $dog_keys[] 'Pola';
      set $dog_keys[] 'Arkas';
      set $dog_keys[] 'Luna';

      iterate $dog_keys;
      do /while _value_;
          put "Key: " _name_ " Value: " _value_ nl;
          next $dogs_keys;
      done;

\end{sfvcode}


\section{Bringing it all together}
The code example on page \pageref{controls example} shows 
everything covered in this chapter in one event
so that it can be easy to see how things work.

\clearpage
\begin{fvcode}{controls example}{Using Procedural controls}
      
proc template;

    define tagset tagsets.controls;

        define event bif;
            put 'Hello Eric' nl /break_if $is_eric;
            put 'Greetings ' operator nl;
        end;

        define event initialize;
            set $is_elmo 'True' /if cmp(operator, 'elmo');
            set $is_eric 'True' /if cmp(operator, 'eric');
            put 'Hello Eric' nl /if $is_eric;
            put 'Greetings ' operator nl /if ^$is_eric;
            put 'RIC' nl /if contains(operator, 'ric');
            put 'This might be Eric' nl /if any(operator, $is_eric);
            put 'This is Eric' nl /if exists(operator, $is_eric);

            trigger bif;

            do /if $is_eric;
                put 'Go away' nl;

            else /if $is_elmo;
                put "What's up?" nl;

            else;
                put 'Hello ' operator ' How are you?' nl;
            done;


            eval $i 0;

            put nl 'i is ' $i nl;

            put "Going into a Loop to 10" nl;
            put "Continue at 5" nl;
            put "stop at 8" nl;

            do /while $i < 10;

                eval $i $i+1;

                continue /if $i eq 5;
                stop     /if $i eq 8;

                put 'I is ' $i nl;

            else;
                put 'do this if i started out > 10' nl;

            done;

            set $dogs['Ernie'] 'Was a great dog';
            set $dogs['Pola'] 'Was really sweet.';
            set $dogs['Arkas'] 'Is a big dog';
            set $dogs['Luna'] 'Is super sweet';

            put nl 'iterating on a dictionary' nl;
            iterate $dogs;
            do /while _value_;
                put "Key: " _name_ " Value: " _value_ nl;
                next $dogs;
            done;
      
            set $dog_keys[] 'Ernie';
            set $dog_keys[] 'Pola';
            set $dog_keys[] 'Arkas';
            set $dog_keys[] 'Luna';
      
            put nl 'iterating on a list' nl;
            iterate $dog_keys;
            do /while _value_;
                put "Key: " _name_ " Value: " _value_ nl;
                next $dog_keys;
            done;

        end;
    end;
run;

ods tagsets.controls operator='eric' file='test.txt';
ods tagsets.controls close;
      
\end{fvcode}


\begin{poutput}{controls_out}{Control statements Example}
Hello Eric
RIC
This might be Eric
This is Eric
Hello Eric
Go away

i is 0
Going into a Loop to 10
Continue at 5
stop at 8
I is 1
I is 2
I is 3
I is 4
I is 6
I is 7

iterating on a dictionary
Key: Ernie Value: Was a great dog
Key: Pola Value: Was really sweet.
Key: Arkas Value: Is a big dog
Key: Luna Value: Is super sweet

iterating on a list
 Value: Ernie
 Value: Pola
 Value: Arkas
 Value: Luna
\end{poutput}

\section{Summary}
The procedural capabilities of tagsets have grown exponentially
with the release of SAS 9.1.  Simple if's are still the most common
and efficient type of if.  But with the addition of where clauses the
capabilities are almost endless.
Single line, immediate, if's are still quite common, breakif enhances 
those abilities for readability and efficiency.  
Do blocks and while provide even more readability, efficiency and control.
