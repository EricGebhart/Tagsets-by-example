\chapter{Using Tagsets with the Libname XML Engine}
The libname XML Engine also uses tagsets.  all of the output generated
from the XML engine is done through a tagset.  The libname engine's 
tagsets differ somewhat from the tagsets used by ODS Markup.  This chapter
will explain how to use tagsets with the libname engine and how they
differ from those used by ODS.

\section{XML Engine vs. ODS}
ODS is more about reporting and
style.  The XML engine is all about data.  Consequently, the tagsets used by
the Engine are much different from those used by ODS.  The tagsets are not
currently interchangeable (SAS 9.1.3) but that is changing.  The XBRL tagset
works with both ODS and the XML engine.  The XML engine may be able to use 
some of the ODS tagsets with reasonable results.  But ODS usually cannot use 
the engine's tagsets.

The first step towards understanding is to run libname with the short\_map tagset.

\begin{sfvcode}
libname mapxml xml 'map.xml'
tagset=tagsets.short_map;
proc copy in=sashelp out=mapxml;
select class;
run;
\end{sfvcode}

The output from this simple job reveals an event model that is not all that
different from th  ods output.  It's just simpler.

\begin{fvoutput}{Engine_map}{The XML Engine's Event Model}
<?xml version="1.0" encoding="iso-8859-1"?>

<doc operator="eric" sasversion="9.1" 
      saslongversion="9.01.01M0D10062003" 
      date="2003-11-17" time="23:08:50" 
      encoding="iso-8859-1" name="CLASS">
   <doc_head>
   </doc_head>
   <doc_body>
      <proc name="CLASS">
         <table name="CLASS">
            <colspecs name="CLASS">
               <colgroup>
                  <colspec_entry name="Name"/>
                  <colspec_entry name="Sex"/>
                  <colspec_entry name="Age"/>
                  <colspec_entry name="Height"/>
                  <colspec_entry name="Weight"/>
               </colgroup>
            </colspecs>
            <table_head name="CLASS">
               <row name="CLASS">
                  <data>
                  </data>
                  <data>
                  </data>
                  <data>
                  </data>
                  <data>
                  </data>
                  <data>
                  </data>
               </row>
            </table_head>
            <table_body name="CLASS">
               <row name="CLASS">
                  <data name="Name" value="Joyce">
                  </data>
                  <data name="Sex" value="F">
                  </data>
                  <data name="Age" value="11">
                  </data>
                  <data name="Height" value="51.3">
                  </data>
                  <data name="Weight" value="50.5">
                  </data>
               </row>
               <row name="CLASS">
                  <data name="Name" value="Thomas">
                  </data>
                  <data name="Sex" value="M">
                  </data>
                  <data name="Age" value="11">
                  </data>
                  <data name="Height" value="57.5">
                  </data>
                  <data name="Weight" value="85">
                  </data>
               </row>
               <row name="CLASS">
                  <data name="Name" value="Louise">
                  </data>
                  <data name="Sex" value="F">
                  </data>
                  <data name="Age" value="12">
                  </data>
                  <data name="Height" value="56.3">
                  </data>
                  <data name="Weight" value="77">
                  </data>
               </row>
               <row name="CLASS">
                  <data name="Name" value="Jane">
                  </data>
                  <data name="Sex" value="F">
                  </data>
                  <data name="Age" value="12">
                  </data>
                  <data name="Height" value="59.8">
                  </data>
                  <data name="Weight" value="84.5">
                  </data>
               </row>
               <row name="CLASS">
                  <data name="Name" value="James">
                  </data>
                  <data name="Sex" value="M">
                  </data>
                  <data name="Age" value="12">
                  </data>
                  <data name="Height" value="57.3">
                  </data>
                  <data name="Weight" value="83">
                  </data>
               </row>
               <row name="CLASS">
                  <data name="Name" value="John">
                  </data>
                  <data name="Sex" value="M">
                  </data>
                  <data name="Age" value="12">
                  </data>
                  <data name="Height" value="59">
                  </data>
                  <data name="Weight" value="99.5">
                  </data>
               </row>
               <row name="CLASS">
                  <data name="Name" value="Robert">
                  </data>
                  <data name="Sex" value="M">
                  </data>
                  <data name="Age" value="12">
                  </data>
                  <data name="Height" value="64.8">
                  </data>
                  <data name="Weight" value="128">
                  </data>
               </row>
               <row name="CLASS">
                  <data name="Name" value="Alice">
                  </data>
                  <data name="Sex" value="F">
                  </data>
                  <data name="Age" value="13">
                  </data>
                  <data name="Height" value="56.5">
                  </data>
                  <data name="Weight" value="84">
                  </data>
               </row>
</table_body>
</table>
</proc>
</doc_body>
</doc>
\end{verbatim}

Using the event\_map tagset with the same job is even more revealing.  Here is a small
section of that output.

\begin{verbatim}
  <row event_name="row" trigger_name="attr_out" section="body" 
       class="Table" id="CLASS" colcount="5" name="CLASS" 
       index="IDX" just="c">

    <data event_name="data" trigger_name="attr_out" section="body" 
       class="Data" id="CLASS" text="COLUMN" value="Jane" 
       colcount="5" col_id="4" name="Name" type="string" 
       index="IDX" just="0">
    </data>
\end{fvoutput}

\section{Advantages of the XML engine}
It is possible to use ODS tagsets with the libname engine, to a point.  They
don't always work very well.  But the XML Libname engine has several tagsets 
created just for it.  They work better, and they do things that ods doesn't
do all that well.  This is mostly because the datastep's relationship with the
XML engine is much closer than it is with ODS.

\subsection{Control options}
There are also some options that allow control over the form the generated xml
will take.  The increased versatility means that the XML engine can create just
about any XML format that you can imagine.

\section{The Tagsets}
The XML engine's tagsets are designed around a single abstract tagset that should
never be used directly, It won't produce much in the way of output if you do.
The design allows most of the tagsets to be very simple.  But understanding it
all is a bit more complicated.

\section{Summary}
When it comes to importing SAS output into spreadsheets there are many choices.
None of them work perfectly.  But some of them work very well.  DDE has been a
popular choice, but using it is somewhat painful.  The excelXP tagset is probably
a better choice for most uses.  The msOffice2K and phtml tagsets also work very well.


