\chapter{Special Cases, Procedures and Operating Systems'}
The examples in this book have shown that there are some problems
when it comes to the Report and Tabulate procedures.  These
problems are surmountable but they do cause some pain.  This
chapter is going to explain the specifics of these problems
and how to work around them.

\section{A Report Procedure problem}
Yet another reason to
convert is that tagsets can be interchanged to create new and 
diffrent outputs.  Usually datastep is used to create HTML, CSV,

\subsection{deferred data}
or DDE to XML,  A different datastep is required for each.  A different
tagset is required for each as well.  The difference is the reusability
and the ease of creation and maintenance.

\section{The Tabulate and Report problem}
More reasons to convert are the availability of style changes, and the
plethora of ODS options that can be used to customize your output.

\subsection{The Table Head section}
sthsthsnt

\subsection{The Table column specifications}
sthsthsnth

\section{HTML on MVS with a PDSE}
When serving web pages that are stored in a PDSE on MVS, the url do not
adhere to the Standards for URI syntax.  Because of this HTML output
that ODS creates will not have links that work.  To fix this there is
a special tagset called MVSHTML that changes the format of the URLS
in the HTML code.  URLS are used in several places, most importantly
in the table of contents, and with the src attribute when including
graphics.  The MVSHTML tagset takes care of all these problems.

\section{Summary}
Through the various examples in this book we have seen the problems
that are caused by the Report and Tabulate procedures.  These problems
are not unsurmountable, but they do complicate the tagsets we create.  
Knowing what the problems are can help save time and frustration when 
creating a tagset.  Or minimize the surprise when a tagset that works 
with other procedures quits working when it comes to using Report or 
Tabulate.


