\chapter{Introduction}
ODS markup and tagsets are the best way to create markup output
from SAS.  This chapter will talk about some of the reasons for
learning tagsets.  As well as what they are and the history 
behind them.

\section{Why Learn Tagsets?}
There are several tagsets provided by SAS for your use.  Everything
from HTML4, Compact HTML, CSV, to Troff and LaTeX. 
So why would you want to learn how they work?  Or how to program
them?  My answer is to ask you several questions.

Are you completely happy with the HTML that ODS generates?
Have you ever wanted to add functionality to any of the markup
destinations generated by ODS?  Have you ever desired a simpler 
LaTeX, or an HTML that includes your company logo and navigation bars?
Have you ever needed to create a file format for data exchange?
Have you written many lines of datastep code to create output?
Have you written datastep code to interact with Excel through DDE?

Tagsets can help you do these things and many more.  The tagset 
solutions will most likely be simpler, and will integrate more 
seamlessly than alternative methods.  Tagsets are also inherently 
more reusable for use in solving other problems.

\section{What are Tagsets?}
\index{GML}
Tagsets are templates, a program, compiled by proc template.  
But tagsets were created to simplify the creation of GML
\footnote{General Markup Language} output.  Because of what they
do, Tagsets are very different from any of the other template 
languages you may already be familiar with.  This chapter will
give a short history of how and why tagsets came about.  This
should help give a broader view and therefore, a better understanding
of how tagsets work. The second part of this chapter will explain
the basics of using proc template with tagsets. 

\section{A short History of Tagsets}
ODS Markup has it's roots in ODS HTML.  I created ODS HTML in 1995.  The internal model 
for ODS HTML was created while keeping other markup languages in
mind, troff and LaTeX in particular.  
The problem with ODS HTML is that it was fixed, the output
only came out one way.  The HTML it created worked well enough most
of the time.  But it could not be everything to everyone.  The HTML
could only change with each release of SAS.  

By 1999 the quickly changing world of HTML was slowing down.  But XML was
looming on the horizon.  Creating a new ODS destination for each flavor of
XML would be impossible if I used the same design as HTML.  

Tagsets solved all of these problems.  Tagsets interface directly to the 
infrastructure of ODS and at the same time, define what should be printed.

The important part was insuring that the features of both the ODS 
interface and the language were rich enough to accomodate future needs.  To
do that, it was necessary to evaluate several different markup languages.

\section{Markup Languages}

First it is necessary to take a short look at a few markup languages what they
have in common, and how they differ.

\subsection{In the Beginning There Was Roff}
\index{roff}
\index{troff}
\index{groff}

Roff was written around 1969 and is a descendent of runoff.  Troff was written in 1973,
and has continued to grow ever since. For our purposes what is important is that troff
is a GML \footnote{Generic Markup Language}.  The other thing that is important is how
different it is to the GML's that are in vogue today.  Troff is not wellformed.  
Commands begin with a '.' and must be at the beginning of a line.  Troff
is not at all like HTML\footnote{Hyper Text Markup Language}.  
Troff does use a gridded table model which is similar to, but more
complicated than HTML.  Like HTML, Troff can also include other
files which can contain style and formatting information.

\subsection{Then there was LaTeX}
\index{tex}
\index{latex}

TeX was written in 1977 by Donald Knuth.  LaTeX which is built upon TeX was written in 
the early 80's by Leslie Lamport.  LaTeX is another GML which is not necessarily wellformed.
It's commands are preceded by '\\\' while attributes and text are commonly surrounded by \{\}.
LaTeX has several table models.  They are more similar to Troff than HTML.  External files
for styles and other inclusions are also common in LaTeX.

\subsection{SGML and friends}
\index{sgml}
\index{xml}
\index{html}
\index{xhtml}
SGML\footnote{Standard Generalized Markup Language} dates from 1969 and was created by Charles F. Goldfarb.
SGML is the basis for HTML and more recently XML and XHTML.  HTML is much more well formed than LaTeX or
Troff.  But it is not necessarily so.  XML is a sort of dumbed down version of SGML.  It retains all
of the basic features but among other things, it is easier to define DTD's 
\footnote{Document Type Definition} and is therefore become much more popular.
XML and XHTML are by definition, required to be wellformed.  
Angle brackets and quotes are the norm for these languages.  More problematic than the format, is
the personal preference for how the HTML is done.  An even greater problem is the growing number
of XML definitions.  Many of the XML definitions do not adhere to basic idea's about tabular data,
which can be the biggest problem of all.   XML and HTML both prefer to have style information in
separate files.

\subsection{Other Formats}
\index{csv}
\index{sylk}
CSV \footnote{Comma Separated Values} 
format is a very simple format which has more in common with flat file formats that have been
used for data exchange for decades.  The important parts of this type of file are the characters
used for field and record separation.  The SYLK \footnote{Symbolic Link} format is another markup
which is not too complex but much different yet again.  These formats do not have the complexity
of the other formats, the tabular data is gridded and there is little, if any, style information.
Although they are simple, they still present problems.  Symbol placement,
the order of the information and it's definitions are issues presented with every new type of markup.

\subsection{What it all means}
Looking at each of these tells us several things.  

\begin{enumerate} 
\item Not all markup uses tags.
\item Tags may or may not be well formed.
\item Tables are usually, but not always, defined as a grid.
\item Writing style information to a separate file is more common than not.
\end{enumerate}

Basically, there are no rules.  For every rule one language follows, there is another that
breaks it.  So the goal became flexibility.  HTML output is the simplest, so that came first.
CSV, came next.  After that, troff, LaTeX and finally XML.  Each format presented new problems
to solve.  More infrastructure was usually needed.  By the time the ODS XML was implemented
Tagsets had proved their versatility.  But with each new tagset there are new 
problems to overcome.  Even now, there are new output formats waiting to stretch the limits
of Tagsets.  

This book will show how tagsets can be stretched in a different way.  
One of the most useful things about tagsets is not the ability to create entirely new output
destinations, But to modify and shape existing output formats to do your bidding.
Creating a tagset from scratch uses all the same skills and is, in some ways, simpler
than modifying existing tagsets.

