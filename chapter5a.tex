\chapter{Trigger Happy}
\index{Events!Triggering}
\index{Statements!Trigger}
\index{Statements!Trigger}
ODS triggers, or requests, many events internally.  
It is also possible to trigger
events from other events. This chapter is going to
explain the trigger statement and how to use it.  

\section{Simple Triggers}
\index{Statements!Trigger}
\index{Events!Triggering}
In previous examples, the trigger statement has been used for simple 
cases where the triggered
event is a simple event without a start or finish.  This type of use
is very straight forward and intuitive.

\index{Variables!Trigger\_name}
There is a special variable that will always have the name of the 
current triggered event.  Trigger\_name will be empty if the current
event was triggered from within ODS.  The event\_name variable will
always have the name of the initial ODS event.  This variable is useful
in explaining how all of this works.

The following example is the simple form of trigger that has been
used previously.  For convenience the doc event is being used for this
example.  The do\_stuff event is being triggered from both the start
and finish of the doc event.  The message from the do\_stuff event 
will print twice, as expected.

\begin{sfvcode}
   define event doc; 
      start:
          trigger do_stuff;
      finish:
          trigger do_stuff;
   end;

   define event do_stuff;
        put 'This is the ' trigger_name ' Event' nl;
   end;

\end{sfvcode}


\section{Events with a state}
\index{Events!state}
\index{Events!Start:}
\index{Events!Finish:}
\index{Variables!State}
Events can have what is called a state.  In the example above,
the doc event has two states.  The first time it is triggered
it's state is start.  The second time it is in the finish state.
The do\_stuff event is stateless.  It does the same thing always.
The 'state' variable indicates the current state.  It's value
will be either start or finish.  

This is important because triggered events preserve the state of
the calling event.  When a trigger statement is in the start section
of an event, the start section of the triggered event is executed. 
Likewise, when the trigger statement is in the finish section, the
finish section of the triggered event is executed.  The following
example defines a start and finish in the triggered event. 
The result is that each message is printed once.

\begin{sfvcode}
   define event doc; 
      start:
          trigger do_stuff;
      finish:
          trigger do_stuff;
   end;

   define event do_stuff;
       start:
           put state ': ' trigger_name ' Event' nl;
       finish:
           put state ': ' trigger_name ' Event' nl;
   end;
\end{sfvcode}

But what if the triggered event only defined a start or finish?  
Then it would only execute for the state section defined.  The following
example will only print one message, once.

\begin{sfvcode}
   define event doc; 
      start:
          trigger do_stuff;
      finish:
          trigger do_stuff;
   end;

   define event do_stuff;
       finish:
           put state ': ' trigger_name ' Event' nl;
   end;
\end{sfvcode}

\index{Statements!Trigger!start}
\index{Statements!Trigger!finish}
\index{Trigger!start}
\index{Trigger!finish}
\index{Triggers!Example}
There are times when this behavior is not desirable.  Events are commmonly
defined with a start and finish that would be placed around another event
or a put.  There is a third argument to the trigger statement that allows
this.  After the event name, start or finish can be specified.  It may be a
good idea to always specify start and finish just to clarify what the code
is doing.  In this example indention is added to help clarify the nesting
of starts and finishes.  
The example in figure \ref{Triggers} on page \pageref{Triggers}
shows all of the various trigger behaviors.  
The output is shown if figure \ref{Triggers_out} on page \pageref{Triggers_out}.

\begin{fvcode}{Triggers}{Triggers Example}

proc template;

   define tagset tagsets.triggers;
       indent = 4;

       define event doc; 
          start:
              trigger do_stuff;
              trigger do_stuff start;
              trigger inbetween;
              trigger do_stuff finish;
          finish:
              trigger do_stuff start;
              trigger inbetween;
              trigger do_stuff finish;
              trigger do_stuff;
       end;

       define event do_stuff;
           start:
               put state ': ' trigger_name nl;
               ndent;
               trigger more_stuff;
           finish:
               trigger more_stuff;
               xdent;
               put state ': ' trigger_name nl;
       end;

       define event more_stuff;
           start:
               put state ': ' trigger_name nl;
       end;

       define event inbetween;
           put state ': ' trigger_name nl;
       end;

   end;
run;

ods tagsets.triggers file='test.txt';
ods tagsets.triggers close;

\end{fvcode}

\begin{fvoutput}{Triggers_out}{Triggers Example}
start: do_stuff
    start: more_stuff
    start: do_stuff
        start: more_stuff
        start: inbetween
    finish: do_stuff
    start: do_stuff
        start: more_stuff
        finish: inbetween
    finish: do_stuff
finish: do_stuff
\end{fvoutput}


\section{summary}
Triggered events are great way to compartmentalize commonly used code.  When combined
with if and breakif they are also a good way to control the program flow.  If your 
events are getting complicated with lots of do / elses then it's probably time to
create some new events and trigger them.  Triggered events are just as fast as
inline code, so performance is not an issue.  Readability and maintenance of the
tagset is.

Previous to SAS 9.1 triggers were the only way to control the program flow.   
Many of the tagsets still have code that reflect this programming style.  Triggered
events are overly complex in some cases. 



