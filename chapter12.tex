\chapter{Extended examples}
This chapter will explore some more complex examples.  The
complexity comes mostly from the combination of multiple
techniques.  From a juggler's point
of view, there are just more balls in the air.  But a ball is
still just a ball.  Any given feature of these new tagsets
is still just as simple as it ever was.  This is one of the
wonderful things about tagsets.  It is easy to reuse and 
combine ideas from different tagsets.

\section{Repeating Headers, and Mirrored Row headers}
When tabular output is extremely wide or long it is often hard to tell
what the data is because the haders have scrolled out of view.  This
tagset solves that problem by putting row headers on both sides of the
table.  It also repeats the headers as in our previous example on page
\pageref{repeat3}

\subsection{The Single Stream Solution}
This solution uses a single stream per row to capture all of the row headers.
This works nicely but when a table has multiple row headers the result isn't
as nice as it could be.  Still, this is an important step towards the next
solution.

\subsection{The Multiple Stream Solution}
This solution uses multiple streams to capture the row headers.  Streams
make everything easy.  The easiest thing would be to use one stream to catch
the rowheaders on each row.  But then, the order would be fixed when a row had
multiple column row headers.  It  would be really nice if the row headers would
be reversed when they are printed on the other end of the table.  To do that
each row header will need it's own stream.

\subsection{The Multiple List Solution}
This solution uses lists to capture the row headers text and alignment.
this is a bit more work but makes it easier to manipulate how the row headers
are rendered.

\section{Automatic Panelling}
In this example we will combine our startpage tagset with some a dictionary 
and some new events to create a tagset that automatically panels output.

\subsection{An Extension of Start Page}
sthsnthsn

\subsection{The solution}
tnhsnthsn

\section{HTML forms}
This example uses as simple proc print to create a simple HTML form with 
option menus created by the proc print.  This is a fairly simple example 
that can be extended to use multiple data sets and procedures to create
more complex HTML forms from ODS.

\subsection{Option Lists}

\subsection{Saving the Lists}

\subsection{Creating the Form}

\section{Summary}
These tagsets are another example of how a basic idea can be
extended to create a nicley flexible output destination.  It can be completely
transparent to the user or it's behavior can be modified at will.


\chapter{A feature Rich Tagset}
In this book we have created many tagsets that can each do something special.
The tagsets have been carefully written to enable those features to be
transparent to anyone who uses them.  It is possible that all of these features
be combined into one tagset.  

\section{Which features?}
two - sided, start page, sliders, table head style, stripes, nobs, byline, web site.

\section{Lining up the inheritance}
Some of the existing tagsets con be used just by changing their inheritance.  If the
order of inheretance is chosen with care then the number of changes can be minimized.

\section{Copy and Paste}
It might just be easier to create one big tagset that does it all...

\section{Macro Variables and Tagset Alias}
Turning all of these things on and off presents some special problems.  Each of them
want macro variables, tagset alias or both.

\section{Summary}
Cool.
