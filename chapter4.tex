\chapter{Creating Variables}
\index{\$}
\index{\$\$}
The set statement has already been used in several examples. 
Set is the primary way to create and modify variables in
tagsets.
Before continuing it would be good to
explain the different types of variables and the methods 
of creating them.  This chapter is dedicated to explaining
the different types of variables that can be created and 
used in tagsets.

The first step towards understanding is to create a tagset that
we can play with.  We know the initialize event happens once and
that it happens at the very beginning.  We can create a framework
that will allow us to play with tagset just by utilizing the the
initialize event.  There are other events we could use.  Doc for
instance.  But initialize will do just fine.  All of the code snippets
can be placed directly into the initialize event in the following code.

\begin{sfvcode}
proc template;
     define tagset tagsets.Variables;

         define event initialize;

             /* put play code here */

         end;
     end;
run;

ods tagsets.variables file='test.txt';
ods tagsets.variables close;

\end{sfvcode}


\section{String Variables}
\index{Statements!Set}
\index{Set}
\index{Statements!UnSet}
\index{UnSet}
The most common variables are string variables.  Any variable
created with the set statement will be a string.  Tagset variables
are variable length, meaning that they are always the size of the
text they contain.  If a variable is assigned an empty string then
it really doesn't exist at all.  But that's ok, because referencing
a non-existant variable is allowed, in fact it's the easiest way
to create conditional values.  Setting a variable to 'true' and unsetting
it for false is the most efficient way to use a variable of this sort.

The set statement always creates strings.  The first argument is
the variable to set.  
All variables that can be created are
prefixed with a \$ or a \$\$.
After that, the set statement works just like a put,  any number
of strings and variables can be specified.  Including the variable
that is being assigned.

The unset statement is used to delete any variables that may have
created.  Unsetting a variable that does not exist does no harm.

\begin{sfvcode}
           set $test 'The value is:' value;           
           set $test2 $test ' and the label is:' label;           

           unset $test;

           put $test2;
\end{sfvcode}

Like the put statement, set can also have datastep functions.  The
functions cannot be nested.

\begin{sfvcode}
           set $test  operator ' ran this job.';           
           set $test2 $test ' On this date: ' date;           
           set $test3 compress($test2, ':.');
\end{sfvcode}

\section{Lists}
\index{Lists}
\index{Lists!Pushing values}
Set can also create variables that are lists.  Lists are denoted by
a suffix of '[]'.  If the brackets are empty the set adds a new entry
to the end of the list.  Lists are 1 based, the first entry always
has an index value of 1.
If there is a number within the brackets the
value at that location in the list will be replaced with the new value.
If the list is not that long, then the value is appended to the end of
the list.  If a list variable is referenced without a subscript the 
return value is the number of entries in the list.

\begin{sfvcode}
            set $test[]  operator ' ran this job.';           
            set $test[] $test[1] ' On this date: ' date;           
            set $test[] compress($test[2], ':.');
\end{sfvcode}

\index{Lists!Negative indexing}
\index{Lists!Popping values}
Lists also accept negative index values.  If the index value is negative
the the list will be indexed from the end instead of the beginning. 
An index of -1 is always the last entry in the list.  Unsetting a
value in a list causes the list to shrink and all entries from that
point on will appear to have moved to eliminate the gap.  The last entry
on a list can be deleted by unsetting the -1 entry.


\begin{sfvcode}
           set $test[] 'Entry 1';
           set $test[] 'Entry 2';
           set $test[] 'Entry 3'; 

           set $test[2] 'The New 2';

           put 'Test has a length of ' $test nl;

           /* print them in reverse order */
           put $test[-1] nl;
           put $test[-2] nl;
           put $test[-3] nl;

           /* delete the first entry */
           unset $test[1];
       
           put 'Test has a length of ' $test nl;
 
           /* delete the whole thing */

           unset $test;

           put 'Test has a length of ' $test nl;

\end{sfvcode}

\section{Dictionaries}

\index{Dictionaries}
Dictionaries are just like lists except that their indexes are strings
instead of numbers.  These are called key's.  In order to create an entry
in a dictionary a key must be provided.

\begin{sfvcode}
            set $test['one']  operator ' ran this job.';           
            set $test['one plus'] $test['one'] ' On this date: ' date;           
            set $test['one sentence'] compress($test['one plus'], ':.') '.';

            put 'Test has a length of ' $test nl;

            set $key 'one sentence';
            put $test[$key] nl;

           unset $test;

\end{sfvcode}

\section{Numeric Variables}
\index{Variables!Numeric}
\index{Numbers}
\index{Numeric Variables}
Numeric variables can be quite useful for list manipulation and many other
things.  Numeric variables can be created with the eval statement.  The eval
statement has two arguments.  The first is the variable to set.  The second
is a where clause.  Eval is one of the few statements that cannot have an
if condition.   The problem arises because / is a valid character in a where clause.
The do statement allows a solution to this problem.
The type of the variable created with the eval statement 
depends on the where clause.  If the result of the clause is numeric then
the variable will be numeric, otherwise it will be a string.  One advantage
of where clauses is that functions can be nested.  But a set statement is
much more efficient, so if a string is desired set is the better choice.

\begin{sfvcode}
      set $key 'one sentence';
      eval $count 1;
      eval $count $count+1;
      eval $position index($test, 's');
      eval $string1 substr($key, $position);
      set $string2 substr($key, $position);
\end{sfvcode}

\section{Stream Variables}
\index{Variables!Stream}
\index{Streams}
\index{Stream Variables}
\index{Statements!Open}
\index{Open}
\index{Statements!Close}
\index{Close}
Occasionally the target output does not fit the event model
used by tagsets.  This is where stream variables are most useful.  Stream
variables are more like a temporary file than a variable.  The set and unset
statements work with stream variables, but it is much more common to create or append
to a stream with the open statement.  Once a stream is open all output will
be directed to the stream.  Streams automatically compress anything that
is written to them. As long as the output is small the stream will
stay in memory.  As it grows it will be written to disk.  A stream is
closed with a close statement.  A stream will be appended to if it is reopened.
Only one stream can be open at a time, so
opening a stream will close any stream that was previously open.  When 
using a stream variable in a put or set or unset the variable is prefixed with a \$\$.

\begin{sfvcode}  
     open deferred;
         put 'we will print this later.  Goodbye.' nl;
     close;
     put 'This will print now.' nl;

     put $$deferred;

     unset $$deferred;

\end{sfvcode}  

\subsection{Stream Specific Statements}
\index{Statements!DelStream}
\index{DelStream}
\index{Statements!PutStream}
\index{PutStream}
There are several stream specific statements besides open and close.  
Stream specific commands don't use the \$\$ prefix because they know that
they are operating on a stream.  Delstream deletes a stream but the command
is redundant because unset does the same thing.  Putstream on the other hand
can be useful.  Putstream writes the given stream directly to the output file.
If another stream is currently open, putstream will reopen that stream when it
is done.  

Another command that is sometimes needed is flush.  Flush forces any buffered output
to be written to the current output stream.  This applies to the 'normal' stream that
points to the output file, or to any other stream which may be open.  
This is sometimes necessary when switching between streams.

\index{Statements!Open!SAS 9.2}
\index{Streams!SAS 9.2}
With the release of SAS 9.2 the open statement will also be able to accept
variables as a stream name.  This is similar to the concept of pointers in
the C programming language.  When a variable is given to an open statement
a stream is opened with the name contained within the variable.
Using list and dictionary variables with the open statement allows for 
easy management of multiple streams within a tagset. 
This simple example behaves just like the previous example.

\begin{sfvcode}  

     set $mystream "deferred";

     open $mystream;
         put 'we will print this later.  Goodbye.' nl;
     close;
     put 'This will print now.' nl;

     put $$deferred;

     unset $$deferred;

\end{sfvcode}  

\section{The Putvars Statment}
\label{putvars_statement}
\index{Statements!Putvars}
\index{Putvars}
putvars has already been used in previous examples to 
examine the variables in an event.
Putvars is a special statement that prints all
the values for a given category of variables.  The first 
argument to putvars is the variable category to print.
The categories are shown in table \ref{Variable Categories}
on page \pageref{Variable Categories}.

\begin{table}\caption{Putvars: Variable Categories}
\label{Variable Categories}
\begin{tabular}{l|l}
%\multicolumn{2}{c}{variable Categories} \\ 
\hline
Event &  Data and Metadata variables\\
Style &  Style variables\\
Memory & Memory variables created with set or eval statements\\
Stream & Stream variables created with the open statement\\
Dynamic & Dynamic variables created within SAS. \\
Lists & Any list variable. \\
Dictionaries & Any Dictionary variable.
\end{tabular}
\end{table}

After the first argument, putvars behaves just like a put,
except that it will execute once for each variable in the
chosen category.  It is an implicit loop where the name of
the variable is placed in a special variable \_name\_, and the
value of that variable is placed in \_value\_.  A very common
use of putvars looks like this.

\begin{sfvcode}
   putvars style _name_ ' : ' _value_ nl;
\end{sfvcode}

\section{Bringing it all together}
It's quite easy to create a tagset to test all of these things.  The initialize
event is perfect for this sort of thing. The doc event should be used with SAS 9.0 or earlier.
The code example on page \pageref{variables example} combines the examples in this chapter 
into one event.  The output is shown in \ref{variables_out} on page \pageref{variables_out}.  

\clearpage
\begin{fvcode}{variables example}{Using Variables}
proc template;

    define tagset tagsets.variables;

        /* define event doc;  */
        define event initialize;

            set $test  operator ' ran this job.';           
            set $test2 $test ' On this date: ' date;           
            set $test3 compress($test2, ':.');

            put 'Simple string Variables' nl;
            putvars memory _name_ ': ' _value_ nl;

            set $test[]  operator ' ran this job.';           
            set $test[] $test[1] ' On this date: ' date;           
            set $test[] compress($test[2], ':.');
         
            put nl 'A List' nl;
            putvars $test _name_ ': ' _value_ nl;

            set $test[] 'Entry 1';
            set $test[] 'Entry 2';
            set $test[] 'Entry 3'; 

            set $test[2] 'The New 2';

            put nl 'The same List' nl;
            put 'Test now has a length of ' $test nl;

            putvars $test _name_ ': ' _value_ nl;

            put nl 'Print the last 3 in reverse' nl;
            /* print them in reverse order */
            put $test[-1] nl;
            put $test[-2] nl;
            put $test[-3] nl;

            /* delete the first entry */
            put 'Deleting test[1]' nl;
            unset $test[1];
       
            put 'Test now has a length of ' $test nl;
 
            putvars $test _name_ ': ' _value_ nl;

            /* delete the whole thing */

            unset $test;

            put 'Test has a length of ' $test nl;

            set $test['one']  operator ' ran this job.';           
            set $test['one plus'] $test['one'] ' On this date: ' date;           
            set $test['one sentence'] compress($test['one plus'], ':.') '.';

            put nl 'A Dictionary' nl;
            put 'Test has a length of ' $test nl;

            putvars $test _name_ ': ' _value_ nl;

            put nl 'Print a dictionary entry with a key' nl;
            set $key 'one sentence';
            put $test[$key] nl;

            unset $test;

            eval $count 1;
            eval $count $count+1;
            eval $position index($key, 's');
            eval $string1 substr($key, $position);
            set $string2 substr($key, $position);

            put nl 'Miscellaneous variables' nl;
            putvars memory _name_ ': ' _value_ nl;

            open deferred;
                put 'we will print this later.  Goodbye.' nl;
            close;
            put nl 'This will print now.' nl;
   
            put $$deferred;
   
            unset $$deferred;

        end;
    end;
run;

ods tagsets.variables file='test.txt';
ods tagsets.variables close;
\end{fvcode}

\begin{poutput}{variables_out}{Using Variables}
Simple string Variables
test: eric ran this job.
test2: eric ran this job. On this date: 2003-10-26
test3: eric ran this job On this date 2003-10-26

A List
: eric ran this job.
: eric ran this job. On this date: 2003-10-26
: eric ran this job On this date 2003-10-26

The same List
Test now has a length of 6
: eric ran this job.
: The New 2
: eric ran this job On this date 2003-10-26
: Entry 1
: Entry 2
: Entry 3

Print the last 3 in reverse
Entry 3
Entry 2
Entry 1
Deleting test[1]
Test now has a length of 5
: The New 2
: eric ran this job On this date 2003-10-26
: Entry 1
: Entry 2
: Entry 3

A Dictionary
Test has a length of 3
one: eric ran this job.
one plus: eric ran this job. On this date: 2003-10-26
one sentence: eric ran this job On this date 2003-10-26.

Print a dictionary entry with a key
eric ran this job On this date 2003-10-26.

Miscellaneous variables
test2: eric ran this job. On this date: 2003-10-26
test3: eric ran this job On this date 2003-10-26
key: one sentence
count: 2
position: 5
string1: sentence
string2: sentence

This will print now.
we will print this later.  Goodbye.
\end{poutput}

\section{Summary}
This chapter uses a fairly simplistic tagset that does
nothing but show the various forms of variables that
can be created.  Strings are by far the most common
type of variable.  Occasionally only a numeric variable
will do what is needed.  It may not be apparent, but 
Dictionaries and lists can be extremely useful for
simplifying problems and creating new possibilities.

