\chapter{Data Step Functions}
\index{Functions}
\index{Data Step!functions}
Data step functions provide important capabilities to tagsets.  Many
solutions are dependent on string manipulation, formatting and 
number conversion.  Without the functions, these solutions would
be very difficult or impossible.  This chapter will explain the usage
of data step functions in tagsets.


\section{Set and Put statements}
\index{Statements!Put}
\index{Statements!Set}
\index{Set}
\index{Put}
The simplest and most efficient use of data step functions is in
the put and set statements.  The only limitation to using functions
in these statements is that they cannot be nested.  This is generally
not a disadvantage.  

Searching through tagsets released by SAS, the most
popular functions are tranwrd, index, substr, scan, strip and
the perl regular expressions.  These functions return values will
be strings or will be converted to strings when used in the set 
and put statements.  



\section{The Eval Statement}
\index{Statements!Eval}
\index{Eval}
If a numeric return value is desired then the function must be used in
an eval statement.  The eval statement also allows for nested functions.
This is because eval supports all valid where clauses which allow this.

\index{Functions!index}
\index{Functions!substr}
\index{Functions!trim}
\index{Functions!strip}
The following example is taken directly from the html tagsets released with SAS 9.1.
This code shows eval being used to capture the location of the spaces in
the url.  Everything else uses a set statement to store string values. It
would be possible to combine multiple statements into one, if the set statements
were replaced with an eval statement that nested the two datastep functions.

This particular event will only happen once per file so that may be a reasonable
thing to do.  But it should be noted that over use of eval and where clauses
can cause tagsets to be slow.

\begin{sfvcode}
        define event urlLoop;
            eval $space_pos index($urlList, " ");

            do /while $space_pos ne 0;

                do /if $space_pos ne 0;
                    set $current_url substr($urlList,1,$space_pos);
                    set $current_url trim($current_url);
                done;

                trigger link;

                set $urlList substr($urlList,$space_pos);
                set $urlList strip($urlList);

                eval $space_pos index($urlList, " ");
            done;
            /* when space_pos is 0 it's either the only link or the last link */
            set $current_url $urlList;
            trigger link;
        end;
\end{sfvcode}

The purpose of this event is to take a stylesheet url that has multiple urls
separated by spaces.  When that occurs, this event will pull apart the 
urls and create multiple link tags in the html.  An ods statement that uses
this feature would look like this.

\begin{verbatim}
       ods html file='test.html' stylesheet='test.css'(url='test.css test2.css');
\end{verbatim}

\index{Functions!scan}
This event is a nice example of several functions, but there is a much
simpler way to do this.  The scan function does all the work of index,
substr, strip, and trim.

\begin{sfvcode}
        define event urlLoop;
            eval $count 1;
            set $current_url scan($urlList, 1, " ");

            /*---------------------------------------------------eric-*/
            /*-- scan returns a space when it fails.                --*/
            /*------------------------------------------------20Nov03-*/
            do /while !cmp($current_url, ' ');

                trigger link;

                eval $count $count + 1
                set $current_url scan($urlList, $count, " ");
            done;
        end;
\end{sfvcode}

\subsection{number conversions}
\index{Functions!inputn}
\index{Functions!input}
\index{Functions!putc}
\index{Statements!Set}
\index{Statements!Eval}
\index{Numbers!Converting}
\index{Strings!Converting to number}
Converting strings to numbers is one of the most common things tagsets need to do.
It seems to happen a lot.  Using numbers to indicate a column
or row in a table is a common desire.  Often these numbers come in the form of a
macro variable.  The inputn function is the way to convert these strings to numbers.
Inputc also works.  But the input function does not.  Using input will result in
a runtime error.  Following code shows common misuse and usage.

\begin{sfvcode}
proc template;
    define tagset tagsets.input;   

        mvar some_number;
    
        define event initialize;
           eval $inputn_value  inputn(some_number, '3.')+5;           
           set $set_inputn  inputn(some_number, '3.');           
           eval $input_value  input(some_number, 3.)+1;           
           put ":" putc("hello", '$', 15) ":" nl;
           put ":" put("hello", '$15') ":" nl;
           putvars mem _name_ ":" _value_ nl; 
        end;
    end;
run;

%let some_number=10;

ods tagsets.input file="input.txt";
ods tagsets.inpuc close;
\end{sfvcode}

The log looks like this:

\begin{sfvoutput}
NOTE: Overwriting existing template/link: Tagsets.Input
NOTE: TAGSET 'Tagsets.Input' has been saved to: SASUSER.TEMPLAT
16         run;
NOTE: PROCEDURE TEMPLATE used (Total process time):
      real time           0.12 seconds
      cpu time            0.12 seconds
      

17         
18         %let some_number=10;
19         
20         ods tagsets.input file="input.txt";
WARNING: Function not found: input
WARNING: Function not found: put
NOTE: Writing TAGSETS.INPUT Body file: input.txt
\end{sfvoutput}

And the Output looks like this.

\begin{sfvoutput}
:hello          :
:
inputn_value:15
set_inputn:10
input_value:11
\end{sfvoutput}

Obviously, some datastep functions do not work.  Input gives a warning but
seems to work anyway.  Put doesn't work at all.  But putc does.  It's probably
a good idea to stay away from functions that give warnings.

\section{advanced usage and debugging}
\index{Statements!Eval}
\index{Statements!Set}
String
manipulation or using inputn to convert a string to a number are easy.  But
sometimes it is not obvious when to use eval or set.  The choice comes down
to the return value of the function and if it is needed as a parameter to 
another function later on.  

\subsection{File I/O}
\label{readfile}
\index{Functions!filename}
\index{Functions!fopen}
\index{Functions!fread}
\index{Functions!fget}
\index{Functions!missing}
\index{Statements!Eval}
\index{Statements!Set}
This next example uses datastep functions to read in a file.  It is important
that some of the return codes be integers so that they can be used as parameters
later on.  The key to understanding this code is using putlog to print the return
values from the functions.  The return values can be very revealing, If the file
is not found the return from the filename function will be missing, or '.'.  if
close is used instead of fclose the return value wil be a large number.  If everything
comes back with 0's all is well.  Except the file identifier, it should be a unique 
number.  It must be an integer to be used by fread and fget functions.  That means 
that eval must be used when the file identifier is created.

\index{Variables!Fixed Length}
Another trick that this example uses is the creation of a variable with a fixed length.
Data step functions always want fixed length variables.  But tagsets do not have any.
The solution is to create a variable full of spaces.  The functions will not change
the length of it, they will only put values in it.

\begin{sfvcode}
/*----------------------------------------------------------------*/
/*-- Reading a file in using datastep functions.  This example  --*/
/*-- comes straight out of the online documentation             --*/
/*-- for fread().                                               --*/
/*----------------------------------------------------------------*/
proc template;
    define tagset tagsets.readfile;
        parent=tagsets.html4;
        embedded_stylesheet=no;

        mvar infile;


        /*-------------------------------------------------------------eric-*/
        /*-- The files should be given as a list with spaces between them.--*/
        /*----------------------------------------------------------8Nov 03-*/
        define event initialize;
            set $filename 'default.txt';

            set $filrf "myfile";
            trigger readfile;
        end;

        define event readfile;
            
            /*--------------------------------------------------------*/
            /*-- Set up the file and open it.                       --*/
            /*--------------------------------------------------------*/
            putlog "Reading in file: " $filename;

            eval $fid 0;

            eval $rc filename($filrf, $filename);
            putlog "Filename Return Code" ":" $rc;
            
            eval $fid fopen($filrf);

            do /if missing($fid);
                putlog "Error: File Not Found, " $filename;
                break;
            done;
                
            putlog "File ID" ":" $fid;

            /*--------------------------------------------------------*/
            /*-- datastep functions  will bind directly to the      --*/
            /*-- variable space as it exists.                       --*/
            /*--                                                    --*/
            /*-- Tagset variables are not like datastep             --*/
            /*-- variables but we can create a big one full         --*/
            /*-- of spaces and let the functions write to it.       --*/
            /*--                                                    --*/
            /*-- This creates a variable that is 200 spaces so      --*/
            /*-- that the function can write directly to the        --*/
            /*-- memory location held by the variable.              --*/
            /*-- in VI, 200i<space>                                 --*/
            /*--------------------------------------------------------*/
            set $file_record  "                                                                                                                                                                                                        ";

            /*--------------------------------------------------------*/
            /*-- Loop over the records in the file                  --*/
            /*--------------------------------------------------------*/
            do /if $fid > 0 ;

                do /while fread($fid) = 0;

                    set $rc fget($fid,$file_record ,200);
                    putlog "FGet rc" ":" $rc;

                    putlog "Record:" trimn($file_record);
                done;
            done;

           /*----------------------------------------------------------*/
           /*-- close up the file.  set works fine for this.         --*/
           /*----------------------------------------------------------*/

            set $rc fclose($fid);
            putlog "close rc" ":" $rc;
            set $rc filename($filrf);
            putlog "filename rc" ":" $rc;

        end;
    end;
run;


ods tagsets.readfile file="file.txt";
ods tagsets.readfile close;


\end{sfvcode}

With simple four line input file, the output to the log generated by 
this tagset looks like this.

\begin{sfvcode}
NOTE: Writing TAGSETS.READFILE Body file: file.txt
Reading in file: default.txt
Filename Return Code:0
File ID:1
FGet rc:0
this is a test
FGet rc:0
the quick brown
FGet rc:0
fox jumped over
FGet rc:0
the lazy dog.
close rc:0
filename rc:0
83         ods tagsets.readfile close;
\end{sfvcode}

\subsection{Perl Regular Expressions}
\index{Perl!Regular Expressions}
\index{Functions!getoption}
\index{Functions!tranwrd}
\index{Functions!lowcase}
\index{Functions!prxmatch}
\index{Functions!prxparse}
\index{Functions!prxposn}
\index{papersize}
\index{Statements!Eval}
\index{Statements!Set}
Another complex example comes from the perl regular expressions.  The compiled
regular expression is a unique integer that is used when getting the substrings from
the matched string.  The following example pulls apart the value of the papersize option
for use in an xml document.  The papersize option is extremely variable in it's format.
That makes the perl regular expressions a good choice for solving this problem. 

\begin{fvcode}{perl_regular_expressions}{Using Perl Regular Expressions}
proc template;
   define tagset tagsets.regex;   
 
     define event initialize;
         set $papersize getoption('PAPERSIZE');

         put "Papersize" ":" $papersize nl;

         /* tranwrd just makes the regex easier. Get rid of optional quotes */
         set $papersize tranwrd($papersize, '"', " ");
         set $papersize tranwrd($papersize, "'", " ");

         /* could be centimeters, could be quoted, or not...
            default is supposedly inches but could be installation
            dependent.
            (8in 11in);
            ('8in', '11in');
            ("8in", "11in");
            ("8", "11");
         */

        /*---------------------------------------------------------------eric-*/
        /*-- The compiled regular expression is a unique integer.           --*/
        /*------------------------------------------------------------21Nov03-*/
         eval $re prxparse('( *([0-9]+) *(IN|CM)* *[,]+ *([0-9]+) *(IN|CM)*.*)') ;
         put "RE is " ":" $re nl;

         eval $match prxmatch($re, $papersize);
         put "MATCH is " ":" $match nl;

        /*---------------------------------------------------------------eric-*/
        /*-- Get the width and the width unit                               --*/
        /*------------------------------------------------------------21Nov03-*/
         set $pwidth prxposn($re, 1, $papersize) ;
         set $pwidth_unit prxposn($re, 2, $papersize) ;

         set $pwidth_unit lowcase($pwidth_unit) /if $pwidth_unit;
         set $pwidth_unit "in" /if !$pwidth_unit;

        /*---------------------------------------------------------------eric-*/
        /*-- Get the height and the height unit.                            --*/
        /*------------------------------------------------------------21Nov03-*/
         set $pheight prxposn($re, 3, $papersize) ;
         set $pheight_unit prxposn($re, 4, $papersize) ;

         set $pheight_unit lowcase($pheight_unit) /if $pheight_unit;
         set $pheight_unit "in" /if !$pheight_unit;

         /* Only if they are non-zero */
         put ' height="' $pheight $pheight_unit '"' nl /if $pheight;
         put ' width="' $pwidth $pwidth_unit '"' nl /if $pwidth;

         unset $papersize;
         unset $re;
         unset $pwidth;
         unset $pwidth_unit;
         unset $pheight;
         unset $pheight_unit;
     end;

 end;
run;

options papersize=("5in" , 10cm);

ods tagsets.regex file="regex.txt";
ods tagsets.regex close;
\end{fvcode}

The output from this tagset looks like this.

\begin{sfvoutput}
Papersize:("5IN" , 10CM)
RE is :1
MATCH is :2
 height="10cm"
 width="5in"
\end{sfvoutput}

\section{Summary}
Date step functions are another versatile tool available to the tagset programmer. 
Functions are simple to use and provide great versatility and power.  Many things 
would not be possible without them.  Functions do need to be used with care, checking
the return values from them is always a good idea.
