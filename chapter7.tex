\chapter{Styles and Tagsets, A perfect match}
\index{Styles}
Understanding how to use ODS styles with tagsets can
simplify many problems.  Styles can help make
a tagset more versatile and reusable.  They can
also help to make the output less verbose and
more compartmentalized. This chapter 
will demonstrate the basics of using styles 
with tagsets.   The best way to start is with
a simple tagset designed specifically
to show how styles work.  


\section{Starting Simple}
The easiest way to examine any problem is to break it down
as much as possible.  It is easy to see how styles work with a
tagset that has one event.  Any event will do. 
The table event is a good place to start.
It is a familiar event, and tables have some
interesting style attributes.  

A single putvars statement can do all the work of revealing how styles work.

\begin{fvcode}{show_style}{Show table style}

proc template;

    define tagset tagsets.showstyle;

        define event table;
            putvars style _name_ ' : ' _value_ nl;
        end;

     end;
run;

ods tagsets.showstyle file='style.txt';
proc print data=sashelp.class; run;
ods tagsets.showstyle close;

\end{fvcode}

The output in figure \ref{show_style_out} shows all the style
attributes for the print procedure's table event.

\begin{poutput}{show_style_out}{The default table style}
FRAME : BOX
RULES : GROUPS
HTMLCLASS : Table
CELLSPACING : 1
CELLPADDING : 7
FRAMEBORDER : auto
CONTENTSCROLLBAR : auto
BODYSCROLLBAR : auto
\end{poutput}

But where did these come from?  ODS has a sytle called default
that it uses if no other style is specified on the ods statement.
The default style has a style element called table which has these
attributes defined.  There is an automatic association between the
table event and the table style.  In fact, most events have a
style that is automatically associated with them.  By default, the
value of HTMLCLASS will be the name of the style element defined in
the style template.

\index{pure\_style}
\index{Attributes!pure\_style}
It is noticable is that there are no colors or fonts, yet
those things must be specified, how could they not be?
In fact, they are defined in the style template.  But they don't show
up here because, by default, tagsets screen out most of the values 
except during a few special events that are used to create style definitons
for the output.  The most common manifestation is an HTML Cascading stylesheet
definition.  If all those style attributes are defined elsewhere then
there is no reason to repeat them on every event.  Setting the pure\_style
attribute to yes will change this behavior so that all the style attributes
are available all of the time.  


It could be argued
that cellpadding and cellspacing belong in the stylesheet,
not here in the table event.  That is technically true.  But some less
capable browsers cannot render tables correctly
using those stylesheet definitions.  So cellpadding and cellspacing remain
here with frame and rules until the web browsers can get it right.  This is
mostly an HTML issue since these things are not a problem in other markup
languages like latex, troff and XML.

\index{Embedded\_stylesheet}
\index{Attributes!Embedded\_stylesheet}
\index{Event!style\_class}
The next step is to create a new style, and add a new event to expose this
behavior.  Figure \ref{show_style2} on page \pageref{show_style2} shows
the new code.  There is a new style called showstyle which contains one
style element.  The table style element defines a font and colors, as
well as some table specific attributes.  The tagset has a new event,
style\_class.  Style\_class is the event used to create each style
definition in the output.  Most events are directed at the body
file by default.  But style\_class is one of those events that is not.
By default style\_class is directed at the stylesheet file.  In order
to see the output from the style\_class event a stylesheet file was specified
on the ods statement.  Instead of adding a stylesheet file, we could have also
just set the embedded\_stylesheet attribute to yes.
The new style must also be specified.  The
output from the body file in figure \ref{show_style2_out} on page \pageref{show_style2_out}
hasn't changed much, it still shows the same variables from before.
But the output in the stylesheet file in figure \ref{show_style2_outb} shows much more.

\begin{fvcode}{show_style2}{Show table style}

proc template;

    define style styles.showstyle;

        style table /
          Font = ("Verdana, Helvetica, sans-serif", 4, bold)
          Foreground = black
          background = White
          bordercolor = green
          cellspacing = 1
          cellpadding = 3
          rules = groups
          frame = box
        ;
    end;

    define tagset tagsets.showstyle;
        indent=4;

        define event style_class;
           trigger put_style;
        end;

        define event table;
           trigger put_style;
        end;

        define event put_style;
            put nl 'Event: ' Event_Name nl;
            ndent;
            putvars style _name_ ' : ' _value_ nl;
            xdent;
        end;

     end;
run;

ods tagsets.showstyle style=showstyle 
                      file='style.txt' 
                stylesheet='style2.txt';

proc print data=sashelp.class; run;
ods tagsets.showstyle close;

\end{fvcode}
       
\begin{poutput}{show_style2_out}{The table event style: style.txt}
Event: table
    FRAME : box
    RULES : groups
    HTMLCLASS : table
    CELLSPACING : 1
    CELLPADDING : 3
    FRAMEBORDER : auto
    CONTENTSCROLLBAR : auto
    BODYSCROLLBAR : auto
\end{poutput}

The stylesheet file contains much more information about the style.

\begin{poutput}{show_style2_outb}{The table class style: style2.txt}
Event: style_class
    FRAME : box
    RULES : groups
    FONT_FACE : Verdana, Helvetica, sans-serif
    HTMLCLASS : table
    CELLSPACING : 1
    CELLPADDING : 3
    FONT_SIZE : medium
    BACKGROUND : #FFFFFF
    FOREGROUND : #000000
    BORDERCOLOR : #008000
    FONT_WEIGHT : bold
    FONT_STYLE : normal
    FONT_WIDTH : normal
    FRAMEBORDER : auto
    CONTENTSCROLLBAR : auto
    BODYSCROLLBAR : auto
\end{poutput}

\subsection{Embedded styles}
\index{Embedded styles}
\index{Stylesheet}
\index{Embedded\_Stylesheet}
\index{Events!Style\_class}
\index{Tagset!inheritance}
\index{inheritance}
Having all of the style definitions in a separate file can be nice.
But there are times when a separate stylesheet file is inconvenient.
This example is one
of those times.  Tagsets allow for this with an attribute called 'embedded\_stylesheet'.
By default the value of this attribute is false.  Turning it on simplifies the
previous example because a separate stylesheet file is no longer needed.  One can
still be specified, and everything will work just like the last example.  But if it isn't,
 hen the style\_class events are directed at the body
file.  In this example inheritance is also used.  This
makes it easy to see what has changed from one tagset to the next.  
For the purposes of this book it also
makes the examples much shorter.  The new tagset is in figure 
\vref{show_style3}.  The corresponding output is
in figure \vref{show_style3_out}.

\begin{fvcode}{show_style3}{Show table style}

proc template;
    define tagset tagsets.showstyle1;
        parent=tagsets.showstyle;
        embedded_stylesheet = yes;
     end;
run;

ods tagsets.showstyle1 style=showstyle file='style.txt';
proc print data=sashelp.class; run;
ods tagsets.showstyle close;
\end{fvcode}
       
Now all the output shows up in the body file.

\begin{poutput}{show_style3_out}{The table event style}
Event: style_class
    FRAME : box
    RULES : groups
    FONT_FACE : Verdana, Helvetica, sans-serif
    HTMLCLASS : table
    CELLSPACING : 1
    CELLPADDING : 3
    FONT_SIZE : medium
    BACKGROUND : #FFFFFF
    FOREGROUND : #000000
    BORDERCOLOR : #008000
    FONT_WEIGHT : bold
    FONT_STYLE : normal
    FONT_WIDTH : normal
    FRAMEBORDER : auto
    CONTENTSCROLLBAR : auto
    BODYSCROLLBAR : auto

Event: table
    FRAME : box
    RULES : groups
    HTMLCLASS : table
    CELLSPACING : 1
    CELLPADDING : 3
    FRAMEBORDER : auto
    CONTENTSCROLLBAR : auto
    BODYSCROLLBAR : auto
\end{poutput}


\section{Styles of your own choosing}       
\index{Attributes!style}
\index{Events!Style\_class}
These examples have been using the style names that ODS decided to use.
But it is possible to create and use style elements using any name.
This can be done by using the the event's style attribute.
The code in figure \ref{show_style4} on page \pageref{show_style4}.  
Adds a new style element, darkbox, and then uses it in the table event.
The output in figure \ref{show_style4_out} on page \pageref{show_style4_out} shows
two style class events and one table event which reveals the new style.  
Note that any new style elements in the ODS style result in another request
of the style\_class event.


\index{Tagset!inheritance}
\index{inheritance}
\begin{fvcode}{show_style4}{Show table style}

proc template;

    define style styles.showstyle;

        style table /
          Font = ("Verdana, Helvetica, sans-serif", 4, bold)
          Foreground = black
          background = White
          bordercolor = green
          cellspacing = 1
          cellpadding = 3
          rules = groups
          frame = box
        ;

        style darkbox /
          Font = ("Times", 6, bold)
          Foreground = black
          background = White
          bordercolor = Black
          cellspacing = 0
          cellpadding = 1
          rules = none
          frame = box
        ;
    end;

    define tagset tagsets.showstyle2;
        parent=tagsets.showstyle1;

        define event table;
           style=darkbox;
           trigger put_style;
        end;
    end;
run;

ods tagsets.showstyle2 style=showstyle file='style.txt';
proc print data=sashelp.class; run;
ods tagsets.showstyle close;
\end{fvcode}
       
\begin{poutput}{show_style4_out}{Specified Style}
Event: style_class
    FRAME : box
    RULES : none
    FONT_FACE : Times
    HTMLCLASS : darkbox
    CELLSPACING : 0
    CELLPADDING : 1
    FONT_SIZE : x-large
    BACKGROUND : #FFFFFF
    FOREGROUND : #000000
    BORDERCOLOR : #000000
    FONT_WEIGHT : bold
    FONT_STYLE : normal
    FONT_WIDTH : normal
    FRAMEBORDER : auto
    CONTENTSCROLLBAR : auto
    BODYSCROLLBAR : auto

Event: style_class
    FRAME : box
    RULES : groups
    FONT_FACE : Verdana, Helvetica, sans-serif
    HTMLCLASS : table
    CELLSPACING : 1
    CELLPADDING : 3
    FONT_SIZE : medium
    BACKGROUND : #FFFFFF
    FOREGROUND : #000000
    BORDERCOLOR : #008000
    FONT_WEIGHT : bold
    FONT_STYLE : normal
    FONT_WIDTH : normal
    FRAMEBORDER : auto
    CONTENTSCROLLBAR : auto
    BODYSCROLLBAR : auto

Event: table
    FRAME : box
    RULES : none
    HTMLCLASS : darkbox
    CELLSPACING : 0
    CELLPADDING : 1
    FRAMEBORDER : auto
    CONTENTSCROLLBAR : auto
    BODYSCROLLBAR : auto
\end{poutput}

\section{Getting the whole style}       
\index{Attributes!pure\_style}
Having a separate place for style definitions is a good thing.  But
sometimes it is necessary to get all the style information all of the
time.  There is another event attribute which allows this to happen.
If pure\_style is set to yes, and the style attribute has been set,
then all attributes of a style will show up in the event.  If this
behavior is desired for all events the pure\_style attribute can be
set as a tagset attribute instead of an event attribute.
In example \vref{show_style5}
pure\_style is set to yes.  The resulting changes to the output from
the table event can be seen in 
figure \vref{show_style5_out}.

\index{Tagset!inheritance}
\index{inheritance}
\begin{fvcode}{show_style5}{Show table style}

proc template;

    define tagset tagsets.showstyle3;
        parent=tagsets.showstyle2;

        define event table;
           style=darkbox;
           pure_style=yes;
           trigger put_style;
        end;
    end;
run;

ods tagsets.showstyle3 style=showstyle file='style.txt';
proc print data=sashelp.class; run;
ods tagsets.showstyle close;
\end{fvcode}

\begin{poutput}{show_style5_out}{Pure style}
Event: style_class
    FRAME : box
    RULES : none
    FONT_FACE : Times
    HTMLCLASS : darkbox
    CELLSPACING : 0
    CELLPADDING : 1
    FONT_SIZE : x-large
    BACKGROUND : #FFFFFF
    FOREGROUND : #000000
    BORDERCOLOR : #000000
    FONT_WEIGHT : bold
    FONT_STYLE : normal
    FONT_WIDTH : normal
    FRAMEBORDER : auto
    CONTENTSCROLLBAR : auto
    BODYSCROLLBAR : auto

Event: style_class
    FRAME : box
    RULES : groups
    FONT_FACE : Verdana, Helvetica, sans-serif
    HTMLCLASS : table
    CELLSPACING : 1
    CELLPADDING : 3
    FONT_SIZE : medium
    BACKGROUND : #FFFFFF
    FOREGROUND : #000000
    BORDERCOLOR : #008000
    FONT_WEIGHT : bold
    FONT_STYLE : normal
    FONT_WIDTH : normal
    FRAMEBORDER : auto
    CONTENTSCROLLBAR : auto
    BODYSCROLLBAR : auto

Event: table
    FRAME : box
    RULES : none
    FONT_FACE : Times
    CELLSPACING : 0
    CELLPADDING : 1
    FONT_SIZE : x-large
    BACKGROUND : #FFFFFF
    FOREGROUND : #000000
    BORDERCOLOR : #000000
    FONT_WEIGHT : bold
    FONT_STYLE : normal
    FONT_WIDTH : normal
    FRAMEBORDER : auto
    CONTENTSCROLLBAR : auto
    BODYSCROLLBAR : auto

\end{poutput}

\index{Variables!style\_element}
\index{Variables!HTMLclass}
Notice that everything is there, except HTMLCLASS.  If the name of
the style is really necessary, there is an event variable called
style\_element which is always populated with the current ods style
element name.

\section{Summary}
This chapter has covered the fundamentals of using ODS styles with
tagsets.  The interaction is really quite simple and direct.  


