\documentclass{book}
\usepackage{listings, fancyvrb, varioref}
\usepackage{fancyhdr, flafter, makeidx, float, times}
\usepackage[T1]{fontenc}
\usepackage{graphicx}
\usepackage{fancyvrb}
\usepackage{listings}
\usepackage{varioref}
\usepackage[color]{ods}
\pagestyle{fancy}

\newenvironment{fvcode}[2]{\label{#1}
 \Verbatim[commandchars=\~\`\`, tabsize=4, xleftmargin=1cm, 
 samepage=false, hfuzz=1pt, fontsize=\small,
 frame=lines, framesep=3mm, label={Code \ref{#1}: \itshape{#2}}]}{\endVerbatim}

\lstnewenvironment{SAScode}[2]
 {\lstset{label={#1}, caption={#2}, frame=lines, xleftmargin=1cm}}
 {}

\begin{document}
\lstset{language=SAS}

\chapter{Testing }
\section{Define statement}
The first thing to do is give a name to the new tagset.  That is done 
with the define statement.  Our tagset is going to be called
plain\_text.  The tagset should be put in proc template's tagset directory,
so it's first name is {\itshape tagsets.} .
The code in Figure \vref{define_tagset}
shows a very simple tagset definition.

\begin{fvcode}{define_tagset}{Defining a Tagset}
     Define tagset tagsets.plain_text;

     end; 
\end{fvcode}

\subsection{A listing example}
This is a very simple example of listing.  The code shown if Figure \vref{SSV}
was created using the listing package.  This example is a direct usage of the
listing environment, I've added arguments to make it look nearly the same as
the fancyvrb example above.  The next example \vref{SSV2} uses the lst environment
created above.  Notice the varioref for \ref{SSV2} adapts nicely to the location
of the listing. The environment makes it easier to use and allows for style changes
in one spot.

\begin{lstlisting}[label=SSV, caption={SSV listing}, frame=lines, xleftmargin=1cm]
proc template;
    
    Define tagset tagsets.SSV;

        parent=tagsets.csv;

        define event header;
            start:
                put ';' / if !cmp(COLSTART, "1");
                put '"';
                put VALUE;
            finish:
                put '"';
        end;

        define event data;
            start:
                put ';' / if !cmp(COLSTART, "1");
                put '"';
                put VALUE;
            finish:
                put '"';
        end;

        define event colspanfill;
            put ';';
        end;    

        define event rowspanfill;
            put ';' /if ! exists(VALUE);
        end;

     end; 
run;

options obs=3;
ods tagsets.ssv file='test.ssv';
proc print data=sashelp.class;run;
ods tagsets.ssv close;
\end{lstlisting}

\begin{SAScode}{SSV2}{SSV listing2}
proc template;
    
    Define tagset tagsets.SSV;

        parent=tagsets.csv;

        define event header;
            start:
                put ';' / if !cmp(COLSTART, "1");
                put '"';
                put VALUE;
            finish:
                put '"';
        end;

        define event data;
            start:
                put ';' / if !cmp(COLSTART, "1");
                put '"';
                put VALUE;
            finish:
                put '"';
        end;

        define event colspanfill;
            put ';';
        end;    

        define event rowspanfill;
            put ';' /if ! exists(VALUE);
        end;

     end; 
run;

options obs=3;
ods tagsets.ssv file='test.ssv';
proc print data=sashelp.class;run;
ods tagsets.ssv close;
\end{SAScode}

\end{document}
