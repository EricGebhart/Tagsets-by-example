
\chapter{Tagsets: how do they work?}
Tagsets are a hybrid of procedural and non-procedural, declarative
programming languages.  A tagset is a collection of event definitions.
Each definition can have procedural elements, but the events themselves
are not procedural.  That's why they are called events.  Events do happen
in some order, but with variation.  They are like the events in your
day.  The alarm clock, brushing your teeth, eating breakfast, going
to work, etc.  They do have some predictability, but are not exactly
the same all the time.  In the context of SAS, events are determined
by the job being run.  There are events for titles and footers, and
page breaks.  There are events for tables and images,  Tables have
varying numbers of observations, columns, and headers.  There are
events for all of those things and more. This chapter will use a 
simple plain text tagset to show how tagsets work.  As the tagset
evolves the output created will reveal the fundamentals of tagset
behavior.

\index{procedural}
\index{Event model}
\section{Data and context in time}

Tagsets are an event driven programming language.
While they do have procedural elements the overall
structure is an event driven model.  Events are best
described as named bundles of data that occur at
different points in time.  The name and timing of
of the event provide context to the data it contains.

\index{Event Requests}
\section{Event Requests}
A tagset defines what is to be done with the data
for any given event.  But it is not necessary to
define all possible events.  Only events which are
meaningful to the output being created
need to be defined.  For this reason, it is more
accurate to think of these data bundles as event
requests.  You may have a title statement in your
SAS job.
When it comes time, ODS will bundle up all the data,
and metadata which defines that title.  The title
event will then be requested.  If the tagset in use has
a definition for the title event then something will
happen.  Otherwise, nothing will happen.

\index{variables}
\subsection{A few variables}
At current count there are 279 variables that could
be defined for any given event.  Most events only
populate a fraction of the available variables.  None
populate all of them at once.


\index{variables!MetaData}
\subsubsection{Data and Metadata}
Of those 279 variables, 172 of them are concerned with
the data and the metadata about the data.  The main variable
for any event is called 'value'.  All of the other variables
are concerned with providing more information abouth that
value.  Name, label, type, missing, rawvalue, justification,
to name a few.  These variables are defined by ODS, the procedure
currently in use, and the table or statgraph template for the
current object.

Still other variables provide context within the output,
the procedure name, the output count, the row, count, the date,
and many more.

\index{variables!Style}
\index{Style}
\subsubsection{style Variables}
The remaining 107 variables are style variables.  These are
the same as the attributes you may have defined in the ODS
style template that you are using.  These variables define things
like font, foreground color, background color, cellspacing, bordercolor,
and many more.

\index{plain\_text}
\section{Our First tagset}
It's best not to get bogged down in the details, or to spend much
time worrying about how all this works.  Tagsets are self revealing,
the best way to learn them is to play with them. Starting with something 
simple is the best way to go.  It will all make sense in time.

This series of examples is going to create a very simple tagset. The output it will create
will be plain text, something like the listing output that SAS has 
generated forever.

\subsection{Define statement}
\index{Define!tagset}
The first thing to do is give a name to the new tagset.  That is done 
with the define statement.  Our tagset is going to be called
plain\_text.  The tagset should be put in proc template's tagset directory,
so it's first name is {\itshape tagsets.} .
The code in listing \vref{define_tagset} shows a very simple tagset definition.

\begin{fvcode}{define_tagset}{Defining a Tagset}
     Define tagset tagsets.plain_text;

     end; 
\end{fvcode}

       
\subsection{The data event}
\index{Define!event}
\index{Event!data}
\index{Table!cell}
The data event is one of the most fundamental events.   This
event is triggered for every value in every observation of a table.  
The define event statement is used to create an event definition.
The code now looks like this \vref{define_event}.

\begin{fvcode}{define_event}{Adding an Event}
     Define tagset tagsets.plain_text;

         Define event data;

         end;

     end; 
\end{fvcode}

\subsubsection{The put statement}
\index{concatenation}
\index{newline}
\index{Put}
\index{Statements!Put}
So far so good but the tagset doesn't do anything yet.  The values in the
data event need to be printed
The put statement is the way to do that.  If you are familiar with data step,
this may seem familiar but it is not quite the same as the put statement in Data Step.
This put automatically concatenates everything.  If you want a newline you must
specify it with nl.  

%\begin{fvcode}{doing_something}{Printing Something}
\begin{lstlisting}
     Define tagset tagsets.plain_text;

         Define event data;
             put 'Data: ' value nl;
         end;

     end; 
\end{lstlisting}
%\end{fvcode}

\subsection{The header event}
\index{Event!header}
\index{Table!cell}
The data will now be printed but there will be no headings.  The
header event will take care of that.  All that is needed is a 
header event that looks just like the data event.

For now, printing each value on a line by it's self is enough.
The working tagset, along with a simple proc print is shown in 
figure \ref{plain_text1} on page \pageref{plain_text1} 
The plain text output is shown in figure \ref{plain_text1_out} on page \pageref{plain_text1_out} 
The corresponding latex output is shown in figure \ref{latex1_out} on page \pageref{latex1_out} 
Notice that the observation count on each row is handled by the header event.

\begin{fvcode} {plain_text1}{A Plain Text Tagset}
 proc template;

     Define tagset tagsets.plain_text;

         Define event data;
             put 'Data: ' value nl;
         end;

         Define event header;
             put 'Header: ' value nl;
         end;


     end; 

 run;

     
 ods latex file='test.tex' stylesheet="test.sty" (url="test");
 ods tagsets.plain_text file='test.txt';
 option obs=3;
 proc print data=sashelp.class; run;
 ods tagsets.plain_text close;
 ods latex close;
\end{fvcode}

\begin{poutput}{plain_text1_out}{Simple Tagset Output}
Header: Obs
Header: Name
Header: Sex
Header: Age
Header: Height
Header: Weight
Header:  1
Data: Alfred
Data: M
Data: 14
Data: 69
Data: 112.5
Header:  2
Data: Alice
Data: F
Data: 13
Data: 56.5
Data: 84.0
Header:  3
Data: Barbara
Data: F
Data: 13
Data: 65.3
Data: 98.0
\end{poutput}

\label{latex1_out}
\sassystemtitle[c]{The SAS System}

\begin{sastable}[c]{rllrrr}\hline
   \multicolumn{1}{|S{Header}{c}|}{Obs} & 
   \multicolumn{1}{|S{Header}{c}|}{Name} & 
   \multicolumn{1}{|S{Header}{c}|}{Sex} & 
   \multicolumn{1}{|S{Header}{c}|}{Age} & 
   \multicolumn{1}{|S{Header}{c}|}{Height} & 
   \multicolumn{1}{|S{Header}{c}|}{Weight}
\\\hline
   \multicolumn{1}{|S{RowHeader}{r}|}{ 1} & 
   \multicolumn{1}{|S{Data}{l}|}{Alfred} & 
   \multicolumn{1}{|S{Data}{l}|}{M} & 
   \multicolumn{1}{|S{Data}{r}|}{14} & 
   \multicolumn{1}{|S{Data}{r}|}{69.0} & 
   \multicolumn{1}{|S{Data}{r}|}{112.5}
\\\hline
   \multicolumn{1}{|S{RowHeader}{r}|}{ 2} & 
   \multicolumn{1}{|S{Data}{l}|}{Alice} & 
   \multicolumn{1}{|S{Data}{l}|}{F} & 
   \multicolumn{1}{|S{Data}{r}|}{13} & 
   \multicolumn{1}{|S{Data}{r}|}{56.5} & 
   \multicolumn{1}{|S{Data}{r}|}{ 84.0}
\\\hline
   \multicolumn{1}{|S{RowHeader}{r}|}{ 3} & 
   \multicolumn{1}{|S{Data}{l}|}{Barbara} & 
   \multicolumn{1}{|S{Data}{l}|}{F} & 
   \multicolumn{1}{|S{Data}{r}|}{13} & 
   \multicolumn{1}{|S{Data}{r}|}{65.3} & 
   \multicolumn{1}{|S{Data}{r}|}{ 98.0}
\\\hline
\end{sastable}

\section{Fleshing out the plain\_text tagset}
\index{Event!Table}
\index{Event!Row}
\index{Event!Data}
\index{Event!Header}
\index{Event!Table\_head}
\index{Event!Table\_body}
\index{Event!Table\_foot}
This is all very nice but to create something more useful
some more events will be needed.  The most basic
events are fairly intuitive.  All tables have a handful
of basic events.  We've seen the data and header events.
The other basic events are, table, row, table\_head,  table\_body, and
 table\_foot.  

The table event encapsulates all other events which define a table.
You may be wondering how an event can contain other
events within it's scope, Or even how an event can have
a scope.

\subsubsection{Start and finish}
\index{Table!row}
\index{Event!start:}
\index{Event!finish:}
Event requests commonly come in matched pairs.  If you are
used to coding HTML then this is a common concept for you.
Most HTML tags are paired as a begin tag and and end tag such
as <span> and </span>.  Tagset events are the same way.  But
to keep things neatly packaged each event definition can hold
both parts.  The beginning and the ending.  In the tagset these
are denoted by start:  and finish:. 

This works because most events get called twice from within ODS.
Once at the beginning and then again at the end.  So when 
print procedure creates a table, the table event is triggered twice, once
at the begininng of the table, and once at the end.  The same
thing goes for the row event, the table\_head, table\_body, and table\_foot events,
even the data and header events.

Our output doesn't look much like a table, but now there are some
more events to try out.  A slight modification to the data and 
header event and the addition of the table and row events will make
the output look much nicer.
The new tagset now looks like the code in
figure \ref{plain_text2} on page \pageref{plain_text2} 

\begin{fvcode}{plain_text2}{Simple Tagset}
 proc template;

     Define tagset tagsets.plain_text;

         Define event table;
             start:
                 put '-----------------------------------------' nl;
             finish:
                 put '-----------------------------------------' nl;
         end;

         Define event row;
             finish:
                 put nl;
         end;

         Define event data;
             put value '   ';
         end;

         Define event header;
             put value ' ';
         end;


     end; 

 run;

     
 ods tagsets.plain_text file='test.txt';
 option obs=3;
 proc print data=sashelp.class; run;
 ods tagsets.plain_text close;
\end{fvcode}

\begin{poutput}{plain_text2_out}{Simple Tagset Output}
-----------------------------------------
Obs Name Sex Age Height Weight 
 1 Alfred   M   14   69   112.5   
-----------------------------------------
\end{poutput}

This version isn't half bad considering the simplicity of the
tagset.  The table event puts lines above and below,  the finish 
of row event causes a newline at the end of each row, and the data
and header values are printed with spaces between them.

\subsection{head, body, and foot}
\index{Table!head}
\index{Table!body}
\index{Table!foot}
\index{Event!table\_head}
\index{Event!table\_body}
\index{Event!table\_foot}
The table\_head, table\_body, and table\_foot events can also be useful 
to us. The head event surrounds the rows that contain the table
headings.  The body section surrounds the data rows, and in the
rare cases that the table has footers at the bottom, the foot section
surrounds those.

For explanation this new tages is going to use all of these events so you can see
where they fit in.  The output will be a bit garish but that can be fixed later.
Our new tagset is shown figure \ref{plain_text3} on page \pageref{plain_text3} 
And the output, expanded to 3 observations is in figure \ref{plain_text3_out} 
on page \pageref{plain_text3_out} 

\begin{fvcode}{plain_text3}{A Simple Tagset}
 proc template;

     Define tagset tagsets.plain_text;

         Define event table;
             start:
                 put '-----------------------------------------' nl;
             finish:
                 put '-----------------------------------------' nl;
         end;

         Define event table_head;
             finish:
                 put '_________________________________________' nl;
         end;

         Define event table_body;
             start:
                 put '.........................................' nl;
             finish:
                 put '.........................................' nl;
         end;

         Define event table_foot;
             start:
                 put '=========================================' nl;
             finish:
                 put '=========================================' nl;
         end;

         Define event row;
             finish:
             put nl;
         end;

         Define event data;
             put value '   ';
         end;

         Define event header;
             put value ' ';
         end;


     end; 

 run;

     
 ods tagsets.plain_text file='test.txt';
 option obs=3;
 proc print data=sashelp.class; run;
 ods tagsets.plain_text close;

\end{fvcode}

\begin{poutput}{plain_text3_out}{Simple Tagset Output}
-----------------------------------------
Obs Name Sex Age Height Weight 
_________________________________________
.........................................
 1 Alfred   M   14   69.0   112.5   
 2 Alice   F   13   56.5    84.0   
 3 Barbara   F   13   65.3    98.0   
.........................................
-----------------------------------------
\end{poutput}

\index{Table!foot}
It is worth noting that the table foot event did not create
any output.  That is because there was no table foot.  If there
had been something inside the table's foot section then that
group of events would have generated output.

\subsection{Titles}
\index{Titles}
\index{Footnotes}
\index{Footers}
\index{Procedure titles}
\index{Note}
\index{Warning}
\index{Error}
\index{Fatal}
\index{Event!system\_title}
\index{Event!system\_footer}
\index{Event!proc\_title}
\index{Event!note}
\index{Event!warning}
\index{Event!error}
\index{Event!fatal}
\index{Event!byline}
There is only one thing that the latex output in figure 
\ref{latex1_out} on page \pageref{latex1_out} has that the plain\_text output doesn't
print the default system title.  Only one more event is needed for that.
But there are other events related to the system\_title event that are worth
noting. There are actually 8 more 
fundamental events which are concerned with titles, and notes.  The names
are self explanatory.  They are, System\_title, System\_footer, Proc\_title,
Byline, Note, Warning, Error, and Fatal.  
Like the incredibly simple data and header events all that is really needed for
each of these is to print the value.  

The new tagset is shown figure \ref{plain_text4} on page \pageref{plain_text4} 
And the output is in figure \ref{plain_text4_out} 
on page \pageref{plain_text4_out} 

\begin{fvcode}{plain_text4}{A Better Simple Tagset}
proc template;

     Define tagset tagsets.plain_text;

         Define event table;
             start:
                 put '-----------------------------------------' nl;
             finish:
                 put '-----------------------------------------' nl;
         end;

         Define event table_head;
             finish:
                 put '_________________________________________' nl;
         end;

         Define event row;
             finish:
             put nl;
         end;

         Define event data;
             put value '   ';
         end;

         Define event header;
             put value ' ';
         end;

         Define event system_title;
             put value nl nl;
         end;

         Define event system_footer;
             put value nl nl;
         end;

         Define event proc_title;
             put value nl nl;
         end;

         Define event note;
             put value nl nl;
         end;

         Define event warning;
             put value nl nl;
         end;

         Define event error;
             put value nl nl;
         end;

         Define event fatal;
             put value nl nl;
         end;

     end; 

 run;

 ods tagsets.plain_text file='test.txt';
 option obs=3;
 proc print data=sashelp.class; run;
 ods _all_ close;
\end{fvcode}

\begin{poutput}{plain_text4_out}{Simple Tagset Output}
The SAS System

-----------------------------------------
Obs Name Sex Age Height Weight 
_________________________________________
 1 Alfred   M   14   69.0   112.5   
 2 Alice   F   13   56.5    84.0   
 3 Barbara   F   13   65.3    98.0   
-----------------------------------------
\end{poutput}

\section{Summary}
By giving away a few events, this chapter has given a
glimpse of how tagsets and events work.  Events can have a start
and finish, and can occur within the scope of another event.  The way
the data event occurs between the start and finish of the row event.

Tagsets can be very simple and so can their events.  This can be used
to advantage.  It is possible to actually write very simple tagsets to explore
how they work.  This is a fundamental concept behind learning, understanding,
and using tagsets.  
