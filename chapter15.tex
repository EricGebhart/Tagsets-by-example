\chapter{Using The HTML Tagsets}
\index{Tagsets!HTML}
\index{Destinations!HTML}
HTML is certainly one the most popular output typs available.
Not only is easily used to report information on the web, but it
is also a commonly understood import format for many applications.
ODS Markup has several HTML tagsets to choose from when it comes
to creating HTML output.  Each of these Tagsets has some advantage
in one way or another depending upon how it will be used.
This chapter will explain the different HTML tagsets and the
features they support.


\section{The Different HTML's}
\index{Destinations!HTML}
ODS HTML is one of the aliased destination names that really 
means use the tagsets.html4 Tagset.  Someday it may well mean
use the Tagsets.html5 Tagset.  But that is in the future.

There are several other aliases, for the various HTML tagsets.
PHTML, HTMLCSS, CHTML and CHTML\_imode are the 
other shortcut names for HTML tagsets.  These aliases map directly
to their tagset names, the aliases just keep us from having to put 'tagsets.'
in front of them.

Only one example is needed in order to
see the major differences between the different HTML tagsets available.
There are differences in style, the way titles and footnotes are processed,
and in the case of the MSOffice2k tagset, there is a difference in how the
tables are rendered.  All that is needed to see this differences is a 
couple of title statements and some simple Proc Tabulate output.

This example has three titles, the default title, and two more titles that
use the 3 part title syntax.  It is the use of style and 3 part titles that
will show major differences between these various HTML destinations.  

\begin{sfvcode}
ods html file="html.html";
ods xhtml file="xhtml.html";
ods htmlcss file="htmlcss.html";
ods phtml file="phtml.html";
ods tagsets.MSOffice2K file="msoffice2k.html";
ods chtml file="chtml.html";
ods imode file="imode.html";


  title3 height=15pt color=orange
         justify=right '15 pt first right'
         justify=left 'first left';
  title4 height=10pt color=red 'no justify red'
         justify=right 'Second right'
         justify=left 'Second left';

proc sort data=sashelp.class out=work.class;
    by age  sex;
run;

options obs=2;

PROC TABULATE DATA=class;
   VAR Height Weight;
   CLASS Sex Age;
   TABLE Age*Mean ALL*Sex*Mean,
         Weight;
   by age;
   
RUN;

ods _all_ close;
\end{sfvcode}


\subsection{HTML4 And XHTML}
\index{Destinations!HTML}
\index{Destinations!HTML4}
\index{Destinations!HTML3}
\index{Destinations!XHTML}
HTML4 and HTML are really the same thing and XHTML is nearly the same but not quite.
The only thing the XHTML tagset does is add XML compliant tags to the HTML.
If you don't know what that is, don't worry about it.  For all practical purposes
they are nearly identical.

The HTML destination does not have to be HTML4, it could be set to be HTML3.
There is a registry setting
that will allow HTML to be the legacy version HTML.  HTML3 is
the name of what was the HTML destination in SAS 8.2.  HTML3 does
not use tagsets or ODS Markup at all.  It is done the old fashioned way
with C-code.  HTML3 has remained virtually unchanged since SAS 8.2.

Let's look at the HTML4 output first since HTML4 is the tagset that does everything.
There aren't any surprises here.  It looks just the way we would expect.
The titles are colored, and justified on the page just the way they should be.

IMAGE MISSING....
%\includegraphics[width=6in]{html.html.png}

Use this output as a reference as you look at the output for the rest of these HTML
destinations.  It will give you a good idea of how they all behave.

\subsection{HTMLCSS}
\index{Destinations!HTMLCSS}
\index{HTMLCSS}
HTMLCSS is a basic stylesheet based HTML tagset.  There is very little difference
between the HTML4 tagset and HTMLCSS. Both the HTML4 and MSOffice2k tagsets inherit
most of their events from the HTMLCSS tagset, and ultimately the PHTML tagset.


\subsection{PHTML}
\index{Destinations!PHTML}
\index{PHTML}
The PHTML tagset generates a very plain HTML.  The huge stylesheet is replaced
by a much smaller stylesheet.  If you dislike all the styles and the verbosity
of the html4 and htmlcss tagsets then this tagset is a good place to start.


\subsection{XHTML}
\index{Destinations!XHTML}
\index{XHTML}
The XHTML tagsets is one of the shortest tagsets around.  It inherits from the HTML4
tagset.  All it adds are the stricter XML style tags.

\subsection{MSOffice2K}
\index{Destinations!MSOffice2k}
\index{MSOffice2k}
The MSOffice2k tagset is a good tagset to use if you want to import the resulting HTML
into Microsoft Office.  This tagset adds Microsoft specific XML to the HTML so that 
Microsoft office can size images better.  It also creates different HTML for titles and
footnotes so that they import more smoothly into Excel.  This tagset also works with proc
tabulate to create a better array of table cells, so that Excel will keep all of the
rows straight.


\subsection{HTML4}
\index{Destinations!HTML!Stylesheets}
The HTML tagsets that use styles always use a stylesheet, if no stylesheet file
is specified the stylesheet is still written to the top of the
body file.  Using an external stylesheet allows more flexibility
and it can allow multiple reports to use the same stylesheet file.

\begin{sfvcode}
   ODS html file='test.html';

   proc print data=sashelp.class; run;

   ODS html close;
\end{sfvcode}



\subsection{PHTML}
\index{Destinations!PHTML}
\index{PHTML}
The PHTML tagset generates a very plain HTML.  The huge stylesheet is replaced
by a much smaller stylesheet.  If you dislike all the styles and the verbosity
of the html4 and htmlcss tagsets then this tagset is a good place to start.


\subsection{HTMLCSS}
\index{Destinations!HTMLCSS}
\index{HTMLCSS}
HTMLCSS is a basic stylesheet based HTML tagset.  There is very little difference
between the HTML4 tagset and HTMLCSS. Both the HTML4 and MSOffice2k tagsets inherit
most of their events from the HTMLCSS tagset, and ultimately the PHTML tagset.

\subsection{XHTML}
\index{Destinations!XHTML}
\index{XHTML}
The XHTML tagsets is one of the shortest tagsets around.  It inherits from the HTML4
tagset.  All it adds are the stricter XML style tags.

\subsection{MSOffice2K}
\index{Destinations!MSOffice2k}
\index{MSOffice2k}
The MSOffice2k tagset is a good tagset to use if you want to import the resulting HTML
into Microsoft Office.  This tagset adds Microsoft specific XML to the HTML so that 
Microsoft office can size images better.  It also creates different HTML for titles and
footnotes so that they import more smoothly into Excel.  This tagset also works with proc
tabulate to create a better array of table cells, so that Excel will keep all of the
rows straight.

\section{HTML and Excel}
\index{Excel!HTML}
\index{Excel!CHTML}
\index{Excel!PHTML}
\index{Excel!MSOffice2k}
\index{HTML}
\index{CHTML}
\index{PHTML}
\index{MSOffice2k}
HTML is one of the better ways to import SAS ouput into excel.  But
some tagsets work better than others.  HTML that uses H tags for 
titles, notes and bylines works better than HTML that uses tables
for their formatting.  That makes the HTML4 and HTMLCSS tagsets 
bad choices for importing into Excel.  The Compact HTML, PHTML, and
the MSOffice2k tagsets work the best. 

\section{CHTML}
\index{Destinations!cHTML}
\index{cHTML}
Compact HTML is a subset of HTML.  Chtml has no style, it has a limited but very
useful set of HTML tags.  Chtml output is viewable on any browser.  It is intended
for use with phones and PDA's where download size is important.

\subsection{CHTML\_imode}
Imode HTML is a subset of Compact HTML.  Imode has no support for tables.  It is
primarily used for phones in Japan.


\section{Stylesheets}
\index{HTML!Stylesheets}
If you choose to use stylesheets it can be convenient to have one stylesheet
that everyone uses.  It is easy to reference another stylesheet on the ODS
Statement.  This first ODS statement will create a stylesheet file called test.css.

\begin{sfvcode}
ODS HTML file='test.html' stylesheet='test.css';
\end{sfvcode}

This next ODS statement does not create a stylesheet file at all.  But it does put 
in a stylesheet link to the test.css stylesheet that was created earlier.

\begin{sfvcode}
ODS HTML file='test.html' stylesheet=(url='test.css');
\end{sfvcode}


\section{Javascript code}
If you have a need for javascript in your HTML there are some good places to put
that code.  There is an event in the tagset called code\_body that will write to
the code file specified on the ods statement.  A link to that file will be created
in the body, contents, and pages files.  It is also possible to trigger the code\_body
event when no code file is specified, that will allow support for both external and 
embedded javascript. 


\section{Accessibility}
The HTML tagsets have accessibility features that are 508 compliant to
level two.  Level 3 compliance is possible with most procedures but 
will require updates to the table templates used by the procedures.
Attributes in the table templates provide for summary, abreviation,
acronym, long\_description, caption, and alt.  Setting these table and
column attributes can improve your HTML's 508 compliance dramatically.  

\section{Summary}
There are several HTML tagsets provided by SAS.  Most likely one of
them will be reasonably close to the style of HTML you wish to create.




